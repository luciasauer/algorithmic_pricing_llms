\documentclass{beamer}
\usepackage[style=apa, backend=biber]{biblatex}
\addbibresource{LaTex/research_proposal/references.bib}
\AtBeginBibliography{\scriptsize} % Reduce size of the bibliography to use less pages
\usepackage{svg}
\usepackage{booktabs}
\usepackage{hwemoji}
\usepackage{graphicx} % Load graphics
\usepackage{booktabs} % Nice tables
\usepackage{dcolumn} % Booktabs column spacing
\usepackage{threeparttable} % Align column caption, table, and notes
\usepackage{adjustbox} % Shrink stuff
\usetheme{metropolis}
\title{Old Questions, New Tools: Revisiting Seminal Economic Research with Causal ML}
\date{\centering\today}
\author{\centering Julian Romero, Lucia Sauer, Moritz Peist}
\institute{\centering\includesvg[width=4cm]{LaTex/imgs/BSE Barcelona Graduate School of Economics.svg}}
\begin{document}

    \begin{frame}{}
        \titlepage
    \end{frame}
    % Add BSE Logo
    \setbeamertemplate{frametitle}
    {
        \nointerlineskip
        \begin{beamercolorbox}[sep=0.3cm,wd=\paperwidth]{frametitle}
            \strut\insertframetitle\strut
            \hfill
            \raisebox{-0.8mm}{\includesvg[width=1cm]{LaTex/imgs/BSE Barcelona Graduate School of Economics.svg}}
        \end{beamercolorbox}
    }

    \begin{frame}
        \frametitle{Outline}
        \tableofcontents
    \end{frame}
    
    \section{Project Overview}
    
    \begin{frame}{Research Question}
        How do causal machine learning approaches affect the \textbf{robustness}, \textbf{precision}, and \textbf{policy implications} of influential economic research compared to traditional econometric methods? 
    \end{frame}
   
    \begin{frame}{Relevance and contribution}
    \textbf{Our idea}: replicate 5 influential economics papers using original methods and re-analyze them using modern causal ML tools.
        \begin{itemize}
            \item Understand and compare main assumptions of each methodology, limitations and advantages.
            \item Compare results, precision, heterogeneous effects and policy implications.
            \item Develop a methodological framework for systematically evaluating when causal ML methods offer advantages over traditional approaches.
            %\item \textcolor{red}{Create accessible tools and visualizations to explain differences in approaches and findings.}
        \end{itemize}

    \end{frame}
   
    \section{Theoretical Framework}

    \begin{frame}{The Two Cultures in Causal Inference}
        \begin{columns}
        \begin{column}{0.48\textwidth}
            \textbf{\centerline{Traditional Econometrics}}
            \begin{itemize}
                \item Emphasis on identification strategy
                \item Simple functional forms
                \item Focus on average effects
                \item Theory-guided specification
                \item Transparent, interpretable models
            \end{itemize}
        \end{column}
        
        \begin{column}{0.48\textwidth}
            \textbf{\centerline{Causal Machine Learning}}
            \begin{itemize}
                \item Flexible functional forms
                \item Heterogeneous treatment effects
                \item Data-driven specification
                \item High-dimensional controls
                \item Cross-validation for robustness
            \end{itemize}
        \end{column}
        \end{columns}
        
        \vspace{0.5cm}
        \centering
        \textit{"The best elements of both cultures are being combined."}\\
        – \textcite{athey2017state}
    \end{frame}
    
    \begin{frame}{Causal ML tools}
    \begin{enumerate}
        \item \textbf{Estimating Heterogeneous Treatment Effects (HTEs)}
            \begin{itemize}
                \item \textbf{Causal Forests} \parencite{athey2018estimation}
                \item \textbf{Generalized Random Forests} \parencite{athey2019generalized}
            \end{itemize}
            
        \item \textbf{Robust Inference in High Dimensions}
            \begin{itemize}
                \item \textbf{Double/Debiased ML (DML)} \parencite{chernozhukov2018double}
            \end{itemize}
            
        \item \textbf{Robust Panel and Time-Series Settings}
            \begin{itemize}
                \item \textbf{Synthetic Difference-in-Differences} \parencite{arkhangelsky2021synthetic}
                \item \textbf{Matrix Completion for Counterfactuals} \parencite{athey2021matrix}
            \end{itemize}
    \end{enumerate}

    \end{frame}

    \begin{frame}{Causal ML packages}
    \begin{figure}
    \centering
    \begin{minipage}{0.45\linewidth}
        \centering
        \includegraphics[width=0.43\linewidth]{LaTex/imgs/milestone1/econml_logo.png}
        \caption{EconML is a Python package for estimating heterogeneous treatment effects from observational data via machine learning, built by the Microsoft Research team.}
        \label{fig:econml}
    \end{minipage}%
    \hfill
    \begin{minipage}{0.5\linewidth}
        \centering
        \includegraphics[width=0.9\linewidth]{LaTex/imgs/milestone1/dowhy_logo.png}
        \caption{Dowhy Package provides a wide variety of algorithms for effect estimation, causal structure learning, diagnosis of causal structures, root cause analysis, interventions and counterfactuals.}
        \label{fig:dowhy}
    \end{minipage}
    \end{figure}

    \end{frame}
    

    \begin{frame}{Selected Papers by Identification Strategy}
        \begin{table}
            \centering
            \tiny
            \begin{tabular}{lll}
                \toprule
                \textbf{Paper} & \textbf{Original Method} & \textbf{Causal ML Alternative} \\
                \midrule
                \textcite{card1994minimum} & Difference-in-Differences & Synthetic DiD, Causal Forests \\
                \addlinespace
                \textcite{angrist2001does} & Regression Discontinuity & RD Forests, GRF \\
                \addlinespace
                \textcite{acemoglu2001colonial} & Instrumental Variables & Double ML for IV \\
                \addlinespace
                \textcite{dehejia1999causal} & Propensity Score Matching & Causal Forests, Ensemble Methods \\
                \addlinespace
                \textcite{autor2020importing} & Staggered DiD with IV & Synthetic DiD, Double ML \\
                \bottomrule
            \end{tabular}
        \end{table}
        
        \vspace{0.3cm}
        \begin{itemize}
            \item Selected papers span multiple identification strategies
            \item Range from classic studies to recent innovations
            \item All have available replication packages
            \item Address policy-relevant questions
        \end{itemize}
    \end{frame}


    \section{Methodology}
    \begin{frame}{Methodology}
        \begin{enumerate}
            \item \textbf{Replication Phase}: Reproduce original study results using authors’ code and data; assessed robustness to minor specification changes.  
            \item \textbf{Causal ML Implementation}: Applied modern causal ML tools (e.g., Double ML, causal forests, RD forests) tailored to each identification strategy; performed hyperparameter tuning and robustness checks.  
            \item \textbf{Comparative Analysis}: Compare estimates, precision, and heterogeneity across methods; evaluated sensitivity and computational trade-offs.  
            \item \textbf{Policy Implications Analysis}: Assess whether ML approaches alter policy conclusions and develop tools to explain and visualize divergences.
        \end{enumerate}
    \end{frame}

    \section{Progress \& Caveats}

    \begin{frame}{Spotlight: \textcite{card1994minimum}}
    \textbf{Minimum Wages and Employment: A Case Study of the Fast-Food Industry}
    
    \begin{table}
        \centering
        \footnotesize
        \begin{tabular}{p{0.45\textwidth}p{0.45\textwidth}}
            \toprule
            \multicolumn{1}{c}{\textbf{Original Approach}} & \multicolumn{1}{c}{\textbf{Our Causal ML Approach}} \\
            \midrule
            Classical DiD design & Synthetic DiD with  features \\
            NJ vs. PA fast food employment & Causal forests for heterogeneity \\
            State border as natural experiment & Chain-specific treatment effects \\
            Linear functional form & Non-parametric functional forms \\
            Homogeneous treatment effect & Spatial heterogeneity analysis \\
            \bottomrule
        \end{tabular}
        \caption{Comparison of methodological approaches for \textcite{card1994minimum}}
        \label{tab:card_krueger_comparison}
    \end{table}
    \begin{center}
    \textit{Key contribution: Understanding when and where minimum wage effects vary}
    \end{center}
    \end{frame}

    \begin{frame}{Caveats}
        \begin{columns}[T]
            \begin{column}{0.48\textwidth}
                \textbf{\centerline{Replication ✅}}
                \begin{minipage}[c][\textheight][c]{\columnwidth}
                    \centering
                    \includegraphics[width=\textwidth]{LaTex/imgs/milestone1/tab4.png}
                    \centerline{vs.}
                    \begin{table}
                        \begin{adjustbox}{width=\textwidth, totalheight=\textheight-2\baselineskip,keepaspectratio}
                            \begin{threeparttable}
                            \caption{Autoscaling table}
                                \input{code/table4_card_krueger}
                            \end{threeparttable}
                        \end{adjustbox}
                    \end{table}
                \end{minipage}
            \end{column}
            \hfill
            \vrule width 0.25pt % Thin vertical rule
            \hfill
            \begin{column}{0.48\textwidth}
                \textbf{\centerline{Causal ML ❎}}
                \begin{minipage}[c][\textheight][c]{\columnwidth}
                    \centering
                    \includesvg[width=\textwidth]{code/causal_ml_treatment_effects.svg}
                    \begin{itemize}
                        \item Few datapoints (357 obs)
                        \item Instability in results
                    \end{itemize}
                \end{minipage}
            \end{column}
        \end{columns}
    \end{frame}
  
    \section{References}
    
    \begin{frame}[allowframebreaks]{References}
        \printbibliography[heading=none]
    \end{frame}
    
\end{document}