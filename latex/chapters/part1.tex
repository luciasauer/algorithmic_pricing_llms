\section{Introduction}\label{sec:intro}

\epigraph{\en{A few years ago, two companies were selling a textbook called The Making of a Fly. One of those sellers used an algorithm which essentially matched its rival's price. That rival had an algorithm which always set a price 27\% higher than the first. The result was that prices kept spiralling upwards, until finally someone noticed what was going on, and adjusted the price manually. By that time, the book was selling – or rather, not selling – for 23 million dollars a copy.}}{Margrethe Vestager, \emph{European Commissioner}, ~\cite*{vestager_algorithms_2017}}

% \begin{displayquote}
%     \en{A few years ago, two companies were selling a textbook called The Making of a Fly. One of those sellers used an algorithm which essentially matched its rival's price. That rival had an algorithm which always set a price 27\% higher than the first. The result was that prices kept spiralling upwards, until finally someone noticed what was going on, and adjusted the price manually. By that time, the book was selling – or rather, not selling – for 23 million dollars a copy.}
%     \begin{flushright}
%         --- Margrethe Vestager, \emph{European Commissioner}, ~\cite*{vestager_algorithms_2017}
%     \end{flushright}
% \end{displayquote}

%\subsection{Background}
%Context (algorithmic pricing, LLMs, FuelWatch)
%\subsection{Motivation}: Why this matters (regulation, digital collusion, real data use)
%\subsection{Research question(s)}: Clear and specific
%\subsection{Contribution}: What's novel in your work
%\subsection{Outline}: What each chapter does

The dawn of artificial intelligence in pricing represents one of the most consequential shifts in market dynamics since the introduction of electronic trading. While Commissioner Vestager's anecdote illustrates the unintended consequences of simple algorithmic interactions, recent advances in Large Language Models (LLMs) suggest far more sophisticated and concerning possibilities. What happens when AI systems capable of strategic reasoning, contextual understanding, and autonomous decision-making are deployed in competitive markets? More critically, do these systems spontaneously develop the exact coordination mechanisms that human competitors have refined over decades of strategic interaction?

\subsection{Background and Motivation}

The rise of algorithmic pricing has fundamentally altered competitive dynamics across industries. From Amazon's dynamic pricing algorithms to airlines' revenue management systems, artificial intelligence increasingly determines the prices consumers pay for goods and services. This technological shift has attracted intense regulatory scrutiny, with competition authorities worldwide expressing concern about the potential of algorithms to facilitate tacit collusion without explicit agreements between firms \parencite{oecd_algorithmic_2023, harrington_developing_2018}. The \textcite{us_department_of_justice_price_2021} emphasizes that unlawful collusion typically requires some form of agreement or coordination among firms. Yet with algorithmic pricing, such coordination may emerge autonomously, complicating legal enforcement.

These concerns are rooted not only in the implications for firm behavior, but in the broader consequences for social welfare: anticompetitive outcomes can lead to higher prices, reduced output, and diminished consumer surplus—hallmarks of inefficiency and welfare loss.

However, most algorithmic collusion research has been confined to simplified theoretical environments or controlled laboratory settings. While \textcite{calvano_artificial_2020} demonstrated that Q-learning algorithms could achieve supracompetitive outcomes in simulated Bertrand competition, and subsequent studies have extended these findings to various market structures, a critical gap remains: we lack evidence of how advanced AI systems behave when confronted with the complexity, uncertainty, and strategic depth of actual market conditions.

This gap is particularly significant given the rapid advancement of Large Language Model technology. Unlike traditional reinforcement learning algorithms, which require long training horizons to converge to stable pricing strategies through trial-and-error exploration—and are vulnerable to adversarial exploitation—LLMs sidestep both concerns. They arrive pre-trained on vast corpora of human-generated text about markets, competition, and strategic behavior, are considerably more robust to manipulation, and have a much lower barrier to entry, as demonstrated by their rapid adoption. As \textcite{fish_algorithmic_2025} demonstrate, this enables LLM agents to recognize and implement sophisticated coordination strategies with speed and effectiveness. Yet their analysis, like virtually all algorithmic collusion research, employs stylized economic environments that abstract away from the rich strategic landscape of real markets.

\subsection{Research Questions and Contributions}

We investigate three fundamental questions at the intersection of artificial intelligence and market competition:

\textbf{Primary Research Question:} Can LLM-based pricing agents autonomously discover and implement the coordination mechanisms documented in real-world gasoline markets when exposed to actual price histories and market conditions?

\textbf{Secondary Questions:}
\begin{enumerate}
    \item How do LLM agents react to cost shocks and dynamic cost environments?
    \item How do LLM agents' coordination patterns compare to the gradual learning processes observed among human competitors?
    \item What role do market structure variables—terminal gate prices, spatial competition, and demand conditions—play in facilitating or constraining LLM coordination?
\end{enumerate}

Our contributions span theoretical, methodological, and policy dimensions. \emph{Theoretically}, we provide the first test of whether algorithmic collusion findings from controlled environments extend to realistic market conditions, addressing a critical external validity question in the growing literature on AI and competition. \emph{Methodologically}, we pioneer the integration of actual market data with LLM agent simulation, establishing a new approach for testing AI behavior under realistic competitive conditions. \emph{From a policy perspective}, our findings generate insights for competition authorities grappling with the regulatory challenges posed by increasingly sophisticated AI pricing systems.

\subsection{Outline}

This thesis proceeds in five additional chapters. \chapterref{sec:litrev} provides a comprehensive literature review spanning algorithmic collusion theory, LLM agent capabilities, empirical coordination studies, and regulatory approaches to AI pricing. \chapterref{sec:meth} details our experimental methodology, data preprocessing procedures, and agent configuration protocols. \chapterref{sec:res} presents our core findings on LLM coordination capabilities and compares agent behavior to documented human coordination patterns. \chapterref{sec:dis} discusses robustness across different market conditions and agent configurations, while \chapterref{sec:con} concludes with policy implications and directions for future research.

% As AI systems become increasingly sophisticated and ubiquitous in competitive markets, understanding their strategic capabilities becomes essential for maintaining competitive market outcomes. Our findings will inform both the development of AI systems and the regulatory frameworks needed to govern their deployment in strategic business contexts.