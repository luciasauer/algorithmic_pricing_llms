\section{Discussion}\label{sec:dis}

Interpretation: What did you observe?

How does it relate to theory and prior work?

Limitations: Data availability, assumptions (no quantities), LLM reproducibility

Policy implications: Transparency, algorithm regulation



%%%%%%%%Just threw some ideas, need to be refined
This thesis has demonstrated that multi-agent pricing strategies powered by Large Language Models (LLMs) can achieve supracompetitive outcomes in oligopolistic market settings. Notably, our results were obtained using significantly smaller and less computationally intensive models than those employed in prior, resource-heavy studies. Despite their relative simplicity, these lightweight models were still capable of producing competitive—and in some cases, strategically sophisticated—pricing behavior.

Our findings highlight several important implications. First, the low barrier to entry for deploying LLMs in economic simulations suggests a promising alternative to traditional reinforcement learning (RL) approaches, which often require extensive tuning and infrastructure. Furthermore, LLMs offer a unique advantage by providing human-interpretable reasoning following each pricing decision, potentially enhancing transparency and trust in automated decision-making systems—something conventional RL methods typically lack.

However, these benefits must be weighed against critical limitations. Most notably, the conclusions drawn from our experiments are not directly generalizable to real-world markets. The stochastic nature of LLMs means outcomes can vary significantly between runs, and their black-box architecture obscures the internal logic behind individual decisions. While the explanations they generate may appear coherent and strategically sound, they may also reflect a form of anthropomorphized bias—textual justifications that are more aesthetically pleasing than truly informative.

In summary, while LLM-based agents represent a promising and accessible tool for modeling strategic economic interactions, their reliability, interpretability, and applicability to real-world markets remain open questions. Further research is needed to systematically evaluate their robustness, improve their transparency, and explore how they might be responsibly integrated into economic policy and practice.
