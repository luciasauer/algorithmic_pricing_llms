\section{Conclusion}\label{sec:con}
% Recap contributions
% 
% Key findings
% 
% Suggestions for future work (e.g., RL agents, including demand shocks, sensitivity/ignorance of prompt to demand function)

This research provides the first systematic evidence that LLM agents, despite their sophisticated coordination capabilities that exceed those documented in traditional Q-learning studies \parencite{calvano_artificial_2020, klein_autonomous_2021}, remain subject to fundamental economic constraints on the sustainability of collusion. While these systems can achieve rapid coordination that surpasses traditional algorithmic approaches, their coordination effectiveness declines predictably as market concentration decreases, validating predictions established by the \emph{Folk Theorem}.

Our findings bridge the gap between theoretical predictions and empirical evidence in algorithmic coordination research. By extending \textcite{fish_algorithmic_2025} duopoly analysis to systematic oligopoly testing, we provide quantitative validation of coordination breakdown patterns that have been theoretically predicted but not empirically tested in AI settings. The 3.7\% per-competitor effect and 10.6\% cumulative breakdown from duopoly to five-agent markets establish concrete benchmarks.

The accessibility and deployment speed of coordination-capable LLMs create both opportunities and challenges for market participants and regulators that extend beyond those identified in traditional algorithmic studies. Understanding the capabilities and limitations of these systems---including their distinctive coordination mechanisms and prompt sensitivity—becomes essential for maintaining competitive market outcomes as AI systems become increasingly prevalent in strategic business applications.

Our findings lay the groundwork for further evidence-based policy approaches to algorithmic coordination, while underscoring the need for ongoing research as AI technologies continue to evolve. The systematic breakdown patterns we document provide grounds for cautious optimism that traditional economic principles remain relevant for governing AI behavior, even as the mechanisms through which these principles operate continue to evolve rapidly with the advancement of AI capabilities.