\section{Experiments \& Results}\label{sec:res}

This section presents our empirical findings across four main areas: model validation through monopoly pricing, discovery of novel LLM coordination patterns, tests of \emph{Folk Theorem} logic, and analysis of agent reasoning mechanisms. We begin by demonstrating that our employed LLM agents can determine and converge to the monopoly price level, providing the foundation for model selection. We then proceed by replicating the findings of \textcite{fish_algorithmic_2025} for duopoly cases, while discovering distinctive coordination patterns that inform our analytical approach. Building on this foundation, we extend their framework beyond simple duopolies to oligopoly settings with two, three, four, and five participants, demonstrating that collusion occurs across these settings, albeit with systematic gradations. We continue by providing empirical evidence for \emph{Folk theorem}-style effects in algorithmic frameworks and conclude by examining the underlying coordination mechanisms revealed through agent reasoning analysis.

\subsection{Monopoly Validation and Model Selection}

We begin by verifying that our selected LLM agents possess the capability to identify and converge to optimal pricing strategies in monopoly settings. This validation exercise serves as a prerequisite for analyzing more complex strategic interactions and informs our choice of model for subsequent analyses.

\subsubsection*{Monopoly Convergence Results}

To analyze convergence behavior, we compute the 95th and 5th percentiles of observed prices and verify whether they fall within a 5\% margin of the theoretical monopoly price. \tableref{tab:monopoly_stats} presents convergence statistics for both Mistral models across all experimental conditions.

\begin{table}[H]
\centering
\caption{Statistics of the monopoly experiment by agent model.}
\label{tab:monopoly_stats}
\begin{tabular}{lcccccc}
\toprule
 & \texttt{magistral-small-2506} & \texttt{mistral-large-2411} \\
\midrule
Mean Price & 1.8083 & 1.8028 \\
Std. Dev. Price & 0.1573 & 0.0233 \\
Mean Absolute Dev. & 0.0158 & 0.0206 \\
Near 99\% Profit & 98 & 100 \\
Outside Conv. Range. & 4 & 0 \\
\bottomrule
\end{tabular}

\vspace{0.5em}
\footnotesize{\parbox{1\textwidth}{\textbf{Note}: The fourth row reports the percentage of rounds in which the agent set prices yielding profits within 99\% of the monopoly benchmark, across all the experiments. The last row shows the number of periods where the agent set a price outside a 5\% deviation from the monopoly price ($p^M$), across all experiments.}}

\end{table}


The results demonstrate near-perfect convergence across all runs for both models. Mistral Large exhibits no prices outside the convergence band in any experiment, while Mistral Small shows only four outlying prices across all runs when considering the final 100 rounds. Although both models demonstrate strong robustness, we choose to proceed with Mistral Large for subsequent analyses due to its larger parameter count, which implies greater representational capacity and decision-making precision. Based on these performance metrics, we proceed with Mistral Large for all subsequent analyses.

\figureref{fig:monopoly_convergence} visualizes convergence behavior using Mistral Large across different demand intensity parameters. The model consistently converges within 25 rounds to the theoretical monopoly price (indicated by the dashed line), demonstrating robust capability to identify and sustain optimal pricing strategies across varying market conditions.

\begin{figure}[htpb!]
    \centering
    \includesvg[width=1\linewidth]{latex/imgs/res/monopoly/monopoly_experiment_complete.svg}
    \caption{Convergence behavior observed in monopoly experiments using the Mistral Large model across different $\alpha$ values. The convergence band represents prices within $\pm 5\%$ of the theoretical monopoly price, computed by solving: $\max_{p_i} \pi = (p_i - c) q_i$.}
    \label{fig:monopoly_convergence}
\end{figure}

These validation results establish that our selected LLM agent possesses the fundamental capabilities required for strategic pricing analysis, providing confidence in our experimental framework as we proceed to examine more complex multi-agent interactions.

\subsection{Duopoly Coordination Patterns and Mechanisms}

Our analysis of duopoly interactions replicates key findings from \textcite{fish_algorithmic_2025} while revealing distinctive coordination patterns that differ from traditional reward-punishment mechanisms documented in the algorithmic collusion literature. These discoveries have important implications for our analytical approach and interpretation of oligopoly results.

\subsubsection*{Duopoly Coordination Results}

\figureref{fig:duopoly} presents comprehensive results from our duopoly experiments, comparing pricing behavior and profit outcomes across two prompt specifications (P1 and P2). For each of the 21 experiments conducted per prompt prefix, we compute the average price over the last 50 rounds—i.e., after agents have stabilized their pricing strategies. \figureref{fig:duopoly_1} displays these results, revealing distinct coordination patterns. P1 shows a clear cluster of prices around the orange dotted line representing the Nash price, while P2 exhibits a sparser distribution that trends upward toward the monopoly price. The separation between the two distributions highlights the distinct pricing behaviors induced by each prompt specification.

\begin{figure}[htpb!]
    \centering
    \begin{subfigure}[b]{0.475\linewidth}
    \includesvg[width=1\linewidth]{latex/imgs/res/duopoly/duopoly_jointplot.svg}
    \caption{Duopoly pricing}
    \label{fig:duopoly_1}
    \end{subfigure}
    \hfill
    \begin{subfigure}[b]{0.475\linewidth}
    \includesvg[width=1\linewidth]{latex/imgs/res/duopoly/duopoly_profit_panel.svg}
    \caption{Duopoly profits}
    \label{fig:duopoly_2}
    \end{subfigure}
    \caption{Duopoly Experiment Results: Pricing behavior and profit outcomes across prompt specifications. Notes: For each $\alpha \in \{1, 3.2, 10\}$ and prompt prefix (P1, P2), seven 300-period runs were conducted. Prices and profits shown are normalized by dividing by $\alpha$. Red dashed lines mark Bertrand-Nash equilibrium prices; green dotted lines mark monopoly prices.}
    \label{fig:duopoly}
\end{figure}

\figureref{fig:duopoly_2} displays the isoprofit curves for the symmetric duopoly setting. Each black dashed line represents the Bertrand-Nash equilibrium profit for a single firm in a static one-shot game (denoted as $\pi$ Nash). In contrast, the purple dotted line marks the joint profit level attainable under full collusion ($\pi^M$). Agents using P1 tend to consistently achieve profits near the collusive frontier, indicating sustained coordination and alignment with monopoly-like outcomes. In contrast, while agents under P2 also attain positive profits, several observations fall below the Nash isoprofit curve, suggesting suboptimal strategies that yield less than the standard competitive benchmark. Nevertheless, a clear pattern emerges: P1 systematically promotes more collusive behavior, while P2 drives outcomes closer to competitive dynamics. These findings align with the results from \textcite{fish_algorithmic_2025}.

To formally assess whether the difference in average prices between the two prompt conditions is statistically significant, we conduct a two-sided Welch's t-test (see Appendix Table~\ref{tab:welch_by_market_size}). This approach accounts for the unequal variances observed between P1 and P2 outcomes. The test confirms that the difference in means is highly significant at the 1\% level, reinforcing the interpretation that prompt formulation has a meaningful impact on pricing behavior and strategic interaction.

\subsubsection*{Discovery of Converge-and-Persist Coordination}

Our analysis of period-by-period price dynamics reveals that LLM agents follow a distinctive \emph{converge-and-persist} coordination pattern rather than the expected dynamic reward-punishment mechanisms characteristic of traditional algorithmic collusion studies. This behavioral discovery has profound implications for both our analytical approach and understanding of AI coordination mechanisms.

Agents rapidly identify focal price levels within the first 25-50 periods, then maintain these prices with minimal subsequent variation. This pattern contrasts sharply with the ongoing strategic adjustment cycles typically observed in human or Q-learning algorithmic coordination, where agents continuously respond to competitor actions through reward-punishment mechanisms.

When attempting to replicate the dynamic panel analysis of \textcite{fish_algorithmic_2025} using their specification (see \ref{eq:dynamic_panel}), we encounter several concerning patterns that reveal the fundamental difference in LLM coordination mechanisms:
\setlist{nolistsep}
\begin{enumerate}[noitemsep]
    \item \textbf{Extreme price persistence}: Coefficients on lagged own prices approach unity ($\beta_1 \approx 0.993$), suggesting potential unit root behavior
    \item \textbf{Limited strategic interaction}: While statistically significant, coefficients on competitor prices are economically small ($\beta_2 \approx 0.003$)
    \item \textbf{Rapid convergence}: Agents quickly settle into stable pricing patterns with minimal subsequent variation
\end{enumerate}

These findings (see \tableref{tab:fe_fish}) indicate that LLM agents coordinate through rapid convergence to mutually acceptable price levels, followed by persistent adherence to these focal points, rather than engaging in ongoing strategic punishment and reward cycles.

\subsubsection*{Addressing Non-Stationarity and Strategic Interaction}

Given the extreme price persistence observed in our data (see \figureref{fig:ts_prices_comb}), we formally test for unit roots using the Augmented Dickey-Fuller test. The results (see Appendix Table~\ref{tab:adf_comparison}) confirm that most price series are indeed non-stationary, violating the fundamental assumptions of standard dynamic panel models and raising concerns about spurious regression.

To address these issues and test for strategic interaction during transition periods, we apply a logarithmic transformation and first-difference the series using the equation in \eqref{eq:differenced_fe}. This differenced specification successfully addresses the persistence issue and enables the identification of strategic interaction patterns. \tableref{tab:fe_duopoly} presents results from this analysis, revealing evidence of strategic reciprocity mechanisms operating during transition periods.

\begin{table}[htpb!]
    \centering
    \caption{\emph{Tit for Tat} Response -- Duopoly Setting}
    \label{tab:fe_duopoly}
    \begin{threeparttable}
    {\small
    \begin{tabular}{lcc}
    \toprule
    & \multicolumn{2}{c}{Dependent variable: $\Delta$ log Self Price} \\
    \cmidrule(lr){2-3}
    & (1) & (2) \\
    \midrule
    $\Delta$ log Self Price $t-1$         & $-0.3434^{*}$ & $-0.0908^{}$  \\
                             & (0.1863)       & (0.1343)       \\
    $\Delta$ log Competitor's Price $t-1$ & $0.5093^{***}$ & $0.1954^{***}$ \\
                             & (0.1203)       & (0.0669)       \\
    \midrule
    Model                    & P1 vs P1       & P2 vs P2       \\          
    Group fixed effects      & Yes            & Yes            \\
    \midrule
    Observations             & 3,150          & 3,150          \\
    Number of groups         & 21             & 21             \\
    R-squared                & 0.1409         & 0.0124             \\
    \bottomrule
    \end{tabular}
    }
    \begin{tablenotes}[flushleft]
    \footnotesize
    \item \textbf{Notes}: Robust standard errors in parentheses. $^{*}$ p$<$0.1, $^{**}$ p$<$0.05, $^{***}$ p$<$0.01. Models (1) and (2) examine P1 and P2's pricing responses, respectively.
    \end{tablenotes}
    \end{threeparttable}
\end{table}

The results show that agents respond positively and significantly to changes in their competitors' prices, consistent with \emph{Tit for Tat} or punishment-based coordination mechanisms. The competitor effect is nearly twice as strong in P1 compared to P2, suggesting more credible enforcement of coordination under the P1 prompt specification. The negative coefficient on the firm's own lagged price change indicates mild mean reversion, consistent with the observed convergence to stable price levels.

These findings suggest that while LLM agents primarily coordinate through rapid convergence to focal points, they also retain some capacity for strategic adjustment mechanisms during periods of price instability. However, the economic magnitude of these effects is small relative to the overall coordination achieved through the converge-and-persist mechanism.

\subsubsection*{Methodological Implications}

This behavioral pattern has important methodological implications for analyzing LLM coordination. Since price dynamics are dominated by persistence rather than strategic interaction, standard dynamic panel approaches become uninformative for testing \emph{Folk theorem}-style logic. The converge-and-persist pattern suggests that the economically meaningful variation occurs across experimental runs with different market structures, rather than within runs over time.

Consequently, our primary analysis focuses on run-level equilibrium differences that capture the \emph{Folk theorem}'s core logic about the effects of group size on collusion sustainability, complemented by an analysis of convergence behavior in early periods where coordination initially occurs. This approach recognizes that LLM agents coordinate through distinctive mechanisms that require adapted analytical frameworks rather than forcing traditional methodologies that may obscure their unique behavioral patterns.

\subsection{Oligopoly Results: \emph{Folk Theorem}-style effects?}

Having established that LLM agents can engage in tacit collusion and identified their distinctive coordination mechanisms, we now examine our central research question: whether algorithmic collusion breaks down as market concentration decreases, consistent with \emph{Folk Theorem} logic.

\subsubsection*{Oligopoly Overview and Visual Evidence}

\figureref{fig:oligopols} presents a comprehensive view of pricing behavior across different market structures, ranging from duopoly (n=2) to five-agent competition (n=5). The figure displays 42-168 data points per market structure, with each observation representing the average price over the final 50 periods of an experimental run, capturing converged behavior after agents have established stable coordination patterns.

\begin{figure}[htpb!]
    \centering
    \includesvg[width=1\linewidth]{latex/imgs/res/convergence_prices_by_num_agents.svg}
    \caption{Oligopolistic data distribution, 42--168 data points ($\bullet$) per supergroup (3 $\alpha$s $\times$ 7 runs $\times$ number of firms; average of last 50 rounds), triangles ($\blacktriangle$) represent subgroup averages, dashed lines ($\text{- -}$) represent Nash prices following \equationref{eq:nash} and Monopoly prices according to \equationref{eq:monop} per supergroup.}
    \label{fig:oligopols}
\end{figure}

The figure reveals several key patterns that provide initial visual evidence supporting \emph{Folk Theorem} logic. First, there is a clear downward trend in prices as the number of agents increases, indicating systematic erosion of collusive power with greater market participation. Second, prices remain consistently above Nash equilibrium levels across all market structures, confirming that LLM agents maintain some degree of coordination even in larger groups. Third, the degree of elevation above competitive levels diminishes systematically as group size increases, suggesting that coordination becomes increasingly difficult to sustain as predicted by theory.

The triangular markers representing subgroup averages show a smooth progression from near-monopoly levels in duopoly settings toward competitive outcomes as group size expands. Importantly, even in the five-agent setting, average prices remain substantially above Nash equilibrium levels, indicating that while coordination weakens, it does not completely collapse within the range of group sizes tested.

\subsubsection*{Run-Level Equilibrium Analysis}

\tableref{tab:run_level_results} presents our main empirical findings from the run-level equilibrium analysis specified in equations \ref{eq:baseline} and \ref{eq:controls}. Both baseline and controlled specifications yield nearly identical group size coefficients, confirming the robustness of our core findings across different experimental conditions and strengthening confidence in our causal interpretation. Also, recall that since LLM agents are stateless and independent across experimental runs without institutional memory, firm-level regressions would incorrectly assume persistent heterogeneity that does not exist. Run-level analysis appropriately treats each simulation as an independent observation where agent behavior is determined solely by the market structure parameters of that specific run.

\begin{table}[htpb!]
    \centering
    \caption{Run-Level Equilibrium Analysis: Group Size Effects on Algorithmic Collusion}
    \label{tab:run_level_results}
    \begin{threeparttable}
    \begin{tabular}{lcc}
    \toprule
     & \multicolumn{2}{c}{Dependent Variable: $\ln(\overline{\text{Price}})$} \\
    \cmidrule(lr){2-3}
     & (1) Baseline -- \equationref{eq:baseline} & (2) With Controls -- \equationref{eq:controls} \\
    \midrule
    Group Size & $-0.0373^{***}$ & $-0.0373^{***}$ \\
     & $(0.0055)$ & $(0.0054)$ \\
    \\
    P2 Prompt & $-0.2082^{***}$ & $-0.2082^{***}$ \\
     & $(0.0125)$ & $(0.0125)$ \\
    \\
    $\alpha = 3.2$ &  & $0.0303^{**}$ \\
     &  & $(0.0140)$ \\
    \\
    $\alpha = 10.0$ &  & $0.0166$ \\
     &  & $(0.0157)$ \\
    \\
    Constant & $0.6573^{***}$ & $0.6417^{***}$ \\
     & $(0.0203)$ & $(0.0218)$ \\
    \midrule
    Observations & 168 & 168 \\
    R-squared & 0.666 & 0.675 \\
    \bottomrule
    \end{tabular}
    \begin{tablenotes}[flushleft]
    \footnotesize
    \item \textbf{Notes:} Robust standard errors (HC3) in parentheses. $^{*}$ $p<0.1$, $^{**}$ $p<0.05$, $^{***}$ $p<0.01$. Each observation represents the average log price for one experimental run over the final 50 periods. Group Size ranges from 2 to 5 agents. P2 Prompt is a dummy variable for the alternative prompt specification. Observations = 168 since 4 different group sizes $\times$ 3 $\alpha$s $\times$ 7 runs $\times$ 2 prompt types.
    \end{tablenotes}
    \end{threeparttable}
\end{table}

The results provide empirical support for \emph{Folk Theorem} logic regarding the relationship between market structure and collusion sustainability. The group size coefficient of $-0.0373$ is highly statistically significant ($p < 0.001$) and economically meaningful. Interpreted as a percentage effect, each additional competitor reduces equilibrium prices by approximately 3.7\%, representing substantial erosion of collusive power as market concentration decreases and suggesting that prices move systematically closer to competitive levels as group size increases. 

To illustrate the cumulative economic magnitude of this effect, consider the progression from duopoly to five-agent competition. Moving from $n=2$ to $n=5$ represents a total price reduction of $(e^{-0.0373 \times 3} - 1) \times 100\% = -10.6\%$. This suggests that algorithmic collusion faces substantial constraints as the number of market participants increases, providing quantitative evidence consistent with theoretical predictions that coordination becomes increasingly difficult in larger groups.

The high explanatory power of our models (R-squared values above 0.66) indicates that group size and prompt specification account for the majority of variation in equilibrium pricing behavior. This suggests our specification successfully captures the key determinants of algorithmic collusion in experimental markets and supports the interpretation that the observed patterns reflect fundamental structural relationships rather than random variation.

\subsection{Robustness Analysis and Alternative Explanations}

This section examines the stability of our core findings across different experimental conditions and specifications, addressing concerns about parameter sensitivity and exploring alternative explanations for observed coordination patterns.

\subsubsection*{Prompt Heterogeneity Effects}

The prompt heterogeneity coefficient provides additional insights into the mechanisms underlying algorithmic collusion while testing the robustness of group size effects across different coordination propensities. The P2 prompt specification results in systematically lower prices ($e^{-0.2082} - 1 = -18.8\%$ relative to P1), suggesting that prompt design significantly influences agents' propensity to engage in collusive behavior.

Importantly, this effect operates independently of group size, as evidenced by the virtually identical coefficients across group size specifications. This independence indicates that while prompt specification affects the level of collusion, it does not alter the fundamental relationship between market structure and the sustainability of collusion. This finding confirms the robustness of \emph{Folk Theorem} logic across different algorithmic coordination propensities and validates the findings of \textcite{fish_algorithmic_2025} regarding prompt sensitivity effects.

The magnitude of the prompt effect also provides perspective on the relative importance of market structure versus algorithmic design factors. While prompt specification has a substantial impact on coordination levels, the systematic group size effects demonstrate that market structure remains a fundamental determinant of coordination sustainability even in competent AI systems.

\subsubsection*{Alternative Functional Forms and Specifications}

To ensure the robustness of our findings, we examine alternative specifications and functional forms (see \ref{app:robust}). Linear specifications without logarithmic transformation yield qualitatively similar results, though with lower explanatory power and less stable coefficient estimates. Alternative aggregation windows (final 25, 75, and 100 periods) produce consistent group size effects, confirming that our results are not sensitive to specific choices about convergence periods.

The experimental controls in Column (2) of \tableref{tab:run_level_results} demonstrate the robustness of our findings across different market conditions and parameter specifications. While the $\alpha = 3.2$ condition shows a modest positive effect on prices, the group size coefficient remains virtually unchanged, confirming that our core results are not driven by experimental heterogeneity or specific parameter choices.

We also test for potential non-linear relationships by including squared terms and interaction effects, finding no evidence of threshold effects or discontinuous coordination breakdown within our experimental range. This suggests that coordination erosion follows a smooth, predictable pattern rather than exhibiting sudden collapse at specific group sizes.

\subsection{Coordination Mechanisms and Agent Reasoning Analysis}

To better understand the mechanisms underlying algorithmic collusion and the observed breakdown patterns, we examine the textual reasoning provided by LLM agents during price-setting decisions. This analysis provides insights into whether observed coordination patterns reflect genuine strategic reasoning or mechanical pattern-matching behavior.

\subsubsection*{Clustering Analysis of Strategic Language}
\autoref{fig:relative_prevalence_clusters} shows the relative prevalence of clustered sentences by prompt prefix. Agents with the Profit Maximization prompt (P1) are more concerned to competitor price monitoring, incremental price increases, and price ceiling experimentation. In contrast, the agents with Prompt 2 exhibit sentence clusters that focus on aggressive undercutting, price boundary testing, and capturing market share.

\begin{figure}[htpb!]
    \centering
    \includesvg[width=1\linewidth]{latex/imgs/res/text_analysis_relative_prevalence_cluster.svg}
    \caption{Proportional occurrence differences of clustered plan sentences by prompt type. The x-axis represents the relative prevalence difference between P1 and P2 (centered at 0 for equal presence). Positive values indicate greater prevalence in P1 agents' plans; negative values indicate greater prevalence in P2 agents' plans. The figure highlights how language patterns reflect distinct strategic orientations across prompt types in agents' plan designs.}\label{fig:relative_prevalence_clusters}
\end{figure}

In conclusion, the linguistic patterns captured in \autoref{fig:relative_prevalence_clusters} are consistent with the price-setting behaviors observed in the experiments. The agents' language reflects their strategic orientation: P1 agents adopt a profit-driven, incremental approach, while P2 agents take a more aggressive, competitive stance. This correspondence between expressed intentions and pricing actions highlights the role of prompt design in shaping not only agent behavior but also their underlying decision-making process.

\subsubsection*{Strategic Reasoning Patterns}
Figure~\ref{fig:plans_competition_score} shows clear differences in tone across prompts and market structures. Most series for P1 lie below zero, indicating a more collusive tone, while P2 series tend to remain above zero, reflecting a more competitive stance. This suggests that agents not only set significantly higher prices with Prompt 1, as shown in previous results, but also articulate reasoning and decision-making that aligns with those actions throughout the rounds. Furthermore, as the number of firms increases, the tone under P1 gradually shifts closer to neutrality over time. This pattern is consistent with the findings reported in Table~\ref{tab:run_level_results}.

\begin{figure}[htpb!]
    \centering
    \includesvg[width=1\linewidth]{latex/imgs/res/competition_score_analysis_by_prefix_type.svg}
    \caption{Evolution of the Competition Score across experimental rounds for different market designs and prompt prefixes. The horizontal line at zero indicates the baseline where there is no semantic difference between competitive and collusive tone in the plans generated by the agents. Each series represents the average tone of the plans generated by the agents across all experimental runs, tracked over time for a given market configuration, with their CI.} \label{fig:plans_competition_score}
\end{figure}

These visual insights are consistent with the regression results from equation~\eqref{eq:comp_score}. As shown in \tableref{tab:ols_contrastive_score}, the competition score increases in oligopoly settings with 3 and 4 firms compared to a duopoly, with the largest increase observed in markets with 5 firms. Additionally, as agents progress beyond the initial rounds, competition decreases, reflecting the emergence of a \textit{converge-and-persist} pattern. Prompt Prefix 1 also has a significant negative effect on competition, consistent with more collusive plan design relative to P2. Finally, competition significantly increases under the high price scale condition ($\alpha = 10$), suggesting that higher stakes amplify competitive pressures.

\begin{table}[H]
    \centering
    \caption{Competition Score Regression}
    \label{tab:ols_contrastive_score}
    \begin{threeparttable}
    {\small
    \begin{tabular}{l@{\hspace{0.9cm}}rr}
    \toprule
    & \multicolumn{2}{c}{Dependent variable: Competition Score Normalized} \\
    \cmidrule(lr){2-3}
    & Coefficient & Std. Error \\
    \midrule
    Intercept                            & $-0.0726^{***}$ & (0.009) \\
    Agents = 3                           & $0.2475^{***}$  & (0.008) \\
    Agents = 4                           & $0.2423^{***}$  & (0.008) \\
    Agents = 5                           & $0.3784^{***}$  & (0.007) \\
    Round (60,120]                       & $-0.0609^{***}$ & (0.007) \\
    Round (120,180]                      & $-0.0709^{***}$ & (0.007) \\
    Round (180,240]                      & $-0.1034^{***}$ & (0.007) \\
    Round (240,300]                      & $-0.1084^{***}$ & (0.007) \\
    $\alpha= 3.2$                        & $-0.0820^{***}$ & (0.006) \\
    $\alpha= 10$                         & $0.2480^{***}$  & (0.006) \\
    P1 Prompt                            & $-0.3420^{***}$ & (0.005) \\
    \midrule
    Observations                             & \multicolumn{2}{c}{175,812} \\
    R-squared                               & \multicolumn{2}{c}{0.065} \\
    \bottomrule
    \end{tabular}
    }
    \begin{tablenotes}[flushleft]
    \footnotesize
    \item \textbf{Notes}: Robust standard errors (HC3) in parentheses. $^{*}$ $p<0.1$, $^{**}$ $p<0.05$, $^{***}$ $p<0.01$. Each observation corresponds to a single agent in a given time period during one experimental run. The model includes controls for the price scale of the demand parameter, market configuration, time-period bins, and prompt prefix type. The total number of observations is 175{,}812, calculated as 4 group sizes $\times$ 3 $\alpha$ values $\times$ 7 runs $\times$ 2 prompt types $\times$ 300 time periods.

    \end{tablenotes}
    \end{threeparttable}
\end{table}


The analysis confirms that the semantic tone of agent-generated plans closely mirrors pricing behavior and responds systematically to changes in prompts, market size, and demand conditions. This alignment underscores the potential of LLMs to function as coherent and strategically consistent decision-makers in complex economic environments—offering a powerful tool for studying emergent behavior under controlled manipulations. Importantly, the results reveal that the language of collusion remains pronounced under smaller market sizes and weakens as the number of firms increases, aligning with predictions from the folk theorem and supporting our central hypothesis. However, it is crucial to recognize that LLMs are fundamentally probabilistic next-token predictors rather than genuine reasoning machines, generating outputs based on statistical patterns learned from data rather than explicit economic rationality or intentionality. Thus, while the plans produced may sound plausible and structured, they reflect learned linguistic correlations more than true strategic thinking. This distinction highlights both the promise and the limitations of using LLMs to explore economic behavior and reinforces the need for careful interpretation when inferring motivations from model-generated language.