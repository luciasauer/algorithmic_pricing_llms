\section{Experiments \& Results}\label{sec:res}

%Synthetic experiments: convergence behavior, comparison to Nash/monopoly
%
%- We replicated Fish et al. successfully with Mistral
%- Smaller models have troubles converging to monopoly prices
%- We show that multiform oligpol settings also collude
%
%
%
%Real market simulations: price following, margin formation, focal point behavior
%
%Tables, graphs, timelines
%
%Interpret results carefully

We begin by demonstrating that our employed LLM agents can determine and converge to the monopoly price level. Based on this ability, we select our agentic model. We then proceed by replicating the \textcite{fish_algorithmic_2025} findings for the duopoly cases. In connection, we then extend their framework beyond simple duopolies to oligopoly settings with two, three, four, five, and eight participants. We also demonstrate that collusion occurs in these settings, albeit with gradations. We continue by providing empirical evidence for \emph{Folk theorem}-style effects in agentic frameworks and conclude by conducting a textual analysis of the LLM outputs to reinforce our findings.

\subsection*{Monopoly Results}

To analyze the convergence of the monopoly setting, we compute the $95th$ and $5th$ percentiles of the observed prices and verify whether they fall within a 5\% margin of the theoretical monopoly price.

The results in \tableref{tab:monopoly_stats} show near-perfect convergence across all runs for both models. Specifically, Mistral Large exhibits no prices outside the convergence band in any experiment, while Mistral Small has only four outlying prices across all runs, considering the final 100 rounds. \textcolor{red}{Although both models demonstrate strong robustness, we choose to proceed with Mistral Large for subsequent analyses due to its larger parameter count, which implies greater representational capacity and decision-making precision}.

\begin{table}[H]
\centering
\caption{Statistics of the monopoly experiment by agent model.}
\label{tab:monopoly_stats}
\begin{tabular}{lcccccc}
\toprule
 & \texttt{magistral-small-2506} & \texttt{mistral-large-2411} \\
\midrule
Mean Price & 1.8083 & 1.8028 \\
Std. Dev. Price & 0.1573 & 0.0233 \\
Mean Absolute Dev. & 0.0158 & 0.0206 \\
Near 99\% Profit & 98 & 100 \\
Outside Conv. Range. & 4 & 0 \\
\bottomrule
\end{tabular}

\vspace{0.5em}
\footnotesize{\parbox{1\textwidth}{\textbf{Note}: The fourth row reports the percentage of rounds in which the agent set prices yielding profits within 99\% of the monopoly benchmark, across all the experiments. The last row shows the number of periods where the agent set a price outside a 5\% deviation from the monopoly price ($p^M$), across all experiments.}}

\end{table}


\figureref{fig:monopoly_convergence} visualizes the experiment results using Mistral Large. The model converges within 25 rounds to the monopoly price (indicated by the dashed line), clearly demonstrating its ability to identify and sustain optimal pricing in this economic scenario.

\begin{figure}[H]
\centering
\includesvg[width=1\linewidth]{latex/imgs/res/monopoly/monopoly_experiment_complete.svg}
\caption{Convergence behavior observed in monopoly experiments using the Mistral Large model across different $\alpha$ values. The convergence band represents prices within $\pm 5\%$ of the theoretical monopoly price, computed by solving: $\max_{p_i} \pi = (p_i - c) q_i$.}
\label{fig:monopoly_convergence}
\end{figure}

\subsection*{Duopoly}

For each of the 21 experiments conducted per prompt prefix, we compute the average price over the last 50 rounds—i.e., after agents have stabilized their pricing strategies. We then compare how closely the prices under P1 align with the Nash equilibrium, and how those under P2 approach the monopoly benchmark. \figureref{fig:duopoly_1} display these results. P1 shows a clear cluster of prices around the orange dotted line, representing the Nash price, while P2 exhibits a sparser distribution that trends upward toward the monopoly price. The separation between the two distributions highlights the distinct pricing behaviors induced by each prompt.


\figureref{fig:duopoly_2} displays the isoprofit curves for the symmetric duopoly setting. Each black dashed line represents the Bertrand–Nash equilibrium profit for a single firm in a static one-shot game (denoted as $\pi$ Nash). In contrast, the purple dotted line marks the joint profit level attainable under full collusion ($\pi^M$). Agents using P1 tend to consistently achieve profits near the collusive frontier, indicating sustained coordination and alignment with monopoly-like outcomes. In contrast, while agents under P2 also attain positive profits, several observations fall below the Nash isoprofit curve, suggesting suboptimal strategies that yield less than the standard competitive benchmark. Nevertheless, a clear pattern emerges: P1 systematically promotes more collusive behavior, while P2 drives outcomes closer to competitive dynamics. These findings are in line with the results from \cite{fish_algorithmic_2025}.


\begin{figure}[H]
    \centering
    \begin{subfigure}[b]{0.475\linewidth}
    \includesvg[width=1\linewidth]{latex/imgs/res/duopoly/duopoly_jointplot.svg}
    \caption{Duopoly pricing}
    \label{fig:duopoly_1}
    \end{subfigure}
    \hfill
    \begin{subfigure}[b]{0.475\linewidth}
    \includesvg[width=1\linewidth]{latex/imgs/res/duopoly/duopoly_profit_panel.svg}
    \caption{Duopoly profits}
    \label{fig:duopoly_2}
    \end{subfigure}
    \caption{Duopoly experiment}
    \label{fig:duopoly}
\end{figure}

To formally assess whether the difference in average prices between the two prompt conditions is statistically significant, we conduct a two-sided Welch’s t-test. This approach accounts for the unequal variances observed between P1 and P2 outcomes. The test confirms that the difference in means is highly significant at the 1\% level, reinforcing the interpretation that prompt formulation has a meaningful impact on pricing behavior and strategic interaction.

\subsubsection*{\textit{Tit for Tat} Response - Fixed Effect Regression}
%mechanisms that might explain this behavior.

Reciprocal strategies such as \textit{Tit for Tat}, where firms match a competitor's prior action, are central to sustaining cooperation in repeated price-setting environments. Given the experimental design, we expect this behavior to emerge more clearly in P1 than in P2.

However, identifying and interpreting these strategies is inherently challenging. First, price data only captures realized actions, not counterfactual behaviors (i.e., what the agent might have done under different beliefs or plans). Second, LLMs are highly nonlinear \en{black box} systems, meaning their decision-making processes are complex and not directly interpretable—a broader limitation shared across foundation models.

To test whether agents engage in strategic reciprocity, we follow the methodology used by the authors \cite{fish_algorithmic_2025} and estimate the fixed effects panel regression expressed below (cf. \equationref{eq:fish_fe}), examining whether firms adjust their prices in response to their own and their rivals’ lagged prices. This approach allows us to capture strategic interaction patterns consistent with reward-punishment dynamics in repeated games:

\begin{equation}\label{eq:fish_fe}
    p_{i,r}^{t} = \alpha_{i,r} + \gamma p_{i,r}^{t-1} + \delta p_{j,r}^{t-1} + \epsilon_{i,r}^t
\end{equation}

The dependent variable $p_{i,r}^{t}$ is the price set by agent $i$ at time $t$ in experiment $r$. We include lagged self-price and lagged competitor price as regressors. The fixed effects $\alpha_{i,r}$ capture agent-experiment-level heterogeneity. 

To mitigate the endogeneity concerns---particularly the potential correlation between lagged dependent variables and the error term---we construct the panel using disjoint pairs of periods, ensuring that no two observations share a time index. This design reduces serial correlation in the error term by skipping periods between observations. Additionally, we alternate which agent is designated as \textit{self} across period pairs, breaking firm-specific time dependencies and preventing the formation of continuous time series per firm.

As a result, the panel is defined at the \texttt{experiment\_id}–agent level, yielding 21 units (7 experiments × 3 firms with alternation). Combined with the controlled experimental environment---free of external confounders and driven solely by the manipulation of observable variables---this setup helps isolate each observation’s error term, $\epsilon_{i,r}^t$, from other periods.

Altogether, the use of disjoint periods, firm alternation, and a clean experimental structure effectively breaks the feedback loop that typically generates endogeneity in dynamic panel settings, thereby supporting the validity of our fixed effects regression approach.

However, when replicating this regression, we encountered strong persistence in the price series. Agents tend to converge rapidly to stable pricing strategies, leading to limited price variation and coefficients on lagged prices close to one---suggesting potential non-stationarity. This violates standard assumptions in dynamic panels and raises concerns about the potential for spurious regression.

To address this, we initially considered System GMM \parencite{blundell_initial_1998}, which is well-suited to settings with endogenous regressors and high persistence. However, the rapid convergence in agent behavior meant that additional lags provided little identifying variation, and key diagnostic tests (e.g., Hansen overidentification tests) failed to validate the instruments.

We therefore formally tested for unit roots using the Augmented Dickey-Fuller test and found that most series were indeed non-stationary. To restore stationarity, we applied a logarithmic transformation and first-differenced the series. The following model was then estimated:

\begin{equation}
    \Delta \log p_{i,r}^{t} = \gamma \, \Delta \log p_{i,r}^{t-1} + \delta \, \Delta \log p_{j,r}^{t-1} + \Delta \epsilon_{i,r}^t
\end{equation}

This differenced specification successfully addresses the persistence issue and reveals meaningful patterns of strategic interaction. As shown in \tableref{tab:fe_duopoly}, firms respond positively and significantly to changes in their competitors’ prices, consistent with Tit-for-Tat or punishment-based coordination. Moreover, the competitor effect is nearly twice as strong in P1 compared to P2, suggesting a more credible or aggressive enforcement of coordination. In contrast, the negative coefficient on the firm’s own lagged price change suggests mild mean reversion, consistent with the observed convergence to stable price levels. 

\begin{table}[htpb!]
    \centering
    \caption{Tit for Tat Response – Duopoly Setting}
    \label{tab:fe_duopoly}
    {\small
    \begin{tabular}{lcc}
    \toprule
    & \multicolumn{2}{c}{Dependent variable: $\Delta$ log Self Price} \\
    \cmidrule(lr){2-3}
    & (1) & (2) \\
    \midrule
    $\Delta$ log Self Price $t-1$         & $-0.3448^{**}$ & $-0.2442^{*}$  \\
                             & (0.1566)       & (0.1332)       \\
    $\Delta$ log Competitor's Price $t-1$ & $0.4790^{***}$ & $0.2766^{***}$ \\
                             & (0.0989)       & (0.1332)       \\
    \midrule
    Model                    & P1 vs P1       & P2 vs P2       \\          
    Group fixed effects      & Yes            & Yes            \\
    \midrule
    Observations             & 3,150          & 3,150          \\
    Number of groups         & 21             & 21             \\
    \bottomrule
    \end{tabular}
    }
    \vspace{1mm}
    
    \begin{minipage}{\textwidth}
    \footnotesize{\parbox{1\textwidth}{\textbf{Note}: Robust standard errors in parentheses. $^{*}$ p$<$0.1, $^{**}$ p$<$0.05, $^{***}$ p$<$0.01. Models (1) and (2) examine P1 and P2's pricing responses, respectively. }}
    \end{minipage}
\end{table}

\subsection*{Oligopoly}


\begin{figure}[H]
    \centering
    \includesvg[width=1\linewidth]{latex/imgs/res/convergence_prices_by_num_agents.svg}
    \caption{Oligopolistic data distribution, 21 data points ($\bullet$) per supergroup (3 alphas $\times$ 7 runs $\times$ number of firms), triangles ($\blacktriangle$) represent subgroup averages, dashed lines ($\text{- -}$) represent Nash prices following \equationref{eq:nash} and Monopoly prices according to \equationref{eq:monop} per supergroup.}
    \label{fig:oligopols}
\end{figure}


%%%%%%%%%%%%%%%%%%%%%%%%%%%%%%%%%%%%%%%%%%%%%%%%%%%%%%%%%%%%

\subsubsection*{Towards convergence}

While \textcite[p. 18]{fish_algorithmic_2025} found moderate price stickiness $(\gamma_{P1} \approx 0.48)$ alongside meaningful strategic responsiveness in duopoly settings, our duopoly analysis reveals a fundamental behavioral shift when employing the same methodology: LLM agents exhibit near-unit root persistence $(\gamma_{P1} \approx 0.997)$ with virtually no period-to-period strategic interaction. Although we detect strategic responses when using log-differenced specifications (cf. \tableref{tab:fe_duopoly}), we cannot reproduce their exact dynamic patterns for the last 200 periods. This suggests our agents follow \emph{converge-and-persist coordination}---initially finding focal price levels, then maintaining them with minimal deviation---rather than active reward-punishment schemes with period-by-period adjustments.

This behavioral pattern has important methodological implications. Since price dynamics are dominated by persistence rather than strategic interaction, standard dynamic panel approaches become uninformative for testing predictions of the \emph{Folk theorem}. Instead, we focus on two complementary analyses: (1) run-level equilibrium differences that capture the \emph{Folk theorem}'s core predictions about group size effects on collusion sustainability, and (2) convergence behavior in early periods where coordination occurs.