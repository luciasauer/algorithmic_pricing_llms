\begin{table}[htpb!]
    \centering
    \caption{Tit for Tat Response}
    \label{tab:csscore_did_results}
    \begin{tabular}{lccccc}
    \toprule
    & \multicolumn{3}{c}{Dependent variable: Complexity Score} \\
    \cmidrule(lr){2-4}
    & (1) & (2) & (3) \\
    \midrule
    Treatment Effect         & 0.084$^{***}$ & 0.059$^{***}$ & 0.073$^{***}$ \\
    & (0.011)       & (0.014)       & (0.010) \\
    &               &               & \\
    \midrule
    Model                    & 2 firms       & 3 firms       & 4 firms & 5 firms \\
    Self $t-1$               & Yes           & Yes           & Yes     & Yes\\
    Group fixed effects      & Yes           & Yes           & Yes     & Yes\\
    Month covariates         & No            & No            & Yes     & Yes\\
    \midrule
    Observations             & 1,328         & 1,328         & 1,328   & 1,328\\
    Number of groups         & 8             & 8             & 8       & 8\\
    \bottomrule
    \multicolumn{4}{p{1\linewidth}}{\footnotesize \textit{Notes:} Standard errors in parentheses, clustered at the group level. $^{*}$ p$<$0.1, $^{**}$ p$<$0.05, $^{***}$ p$<$0.01. The dependent variable is the standardized complexity score calculated at the individual question level. Each question's complexity is measured as the average of four z-standardized components: tag count, code length, body length, and title length. Treatment is defined as the period after ChatGPT's release (November 30, 2022). Models 1-2 use traditional DiD specifications, while Model 3 uses synthetic control methods with month covariates.} \\
    \end{tabular}
\end{table}
