\appendix

\section{Appendix}

\subsection*{Duopoly}

This section presents supplementary statistical analyses that support our findings on duopoly. We provide formal statistical tests comparing prompt specifications, stationarity analysis of price series, and an attempted replication of the core empirical framework from \textcite{fish_algorithmic_2025}.
%%%%%%%%%%%%%%%%%%%%%%%%%%%%%%%%%%%%%%%%%%%%%%%%%%%%%%%%%%%%%%%%%%%%%%%
%Welch Ttest for unequal variances
\begin{table}[H]
    \centering
    \caption{Welch's t-test: Mean Prices Across Prompt Prefixes by Market Size}
    \label{tab:welch_by_market_size}
    \begin{threeparttable}
    \begin{tabular}{lcccc}
    \toprule
    Market Size & Mean P1 & Mean P2 & Welch's t-statistic & p-value \\
    \midrule
    2 Agents & 1.768 & 1.478 & 7.423 & $<$0.001$^{***}$ \\
    3 Agents & 1.741 & 1.423 & 7.774 & $<$0.001$^{***}$ \\
    4 Agents & 1.706 & 1.315 & 8.820 & $<$0.001$^{***}$ \\
    5 Agents & 1.637 & 1.304 & 7.859 & $<$0.001$^{***}$ \\
    \bottomrule
    \end{tabular}
    \begin{tablenotes}[flushleft]
    \footnotesize
    \item \textbf{Notes:} Welch’s t-tests compare the average prices between Prompt 1 (P1) and Prompt 2 (P2) across different market sizes (2 to 5 agents), assuming unequal variances. Each condition includes 21 observations per group. $^{*}$ p$<$0.1, $^{**}$ p$<$0.05, $^{***}$ p$<$0.01.
    \end{tablenotes}
    \end{threeparttable}
\end{table}


%%%%%%%%%%%%%%%%%%%%%%%%%%%%%%%%%%%%%%%%%%%%%%%%%%%%%%%%%%%%%%%%%%%%%%%
%Table of Augmented DF Test
\begin{table}[H]
    \centering
    \caption{ADF Test Summary: Stationarity in Price vs. $\Delta \log(\text{Price})$ Series}
    \label{tab:adf_comparison}
    \begin{threeparttable}
    \begin{tabular}{lcc}
    \toprule
    & Price Series & $\Delta \log(\text{Price})$ \\
    \midrule
    Stationary (p $<$ 0.05)     & 37 (44.0\%)  & 79 (94.0\%) \\
    Non-Stationary (p $\geq$ 0.05) & 47 (56.0\%)  & 5 (6.0\%) \\
    \midrule
    Total Tested Series         & 84           & 84 \\
    \bottomrule
    \end{tabular}
    \begin{tablenotes}[flushleft]
    \footnotesize
    \item \textbf{Notes:} ADF tests were conducted on 84  experiment and firm (\texttt{run\_firm\_id}) level price series. While most raw series fail to reject the null of a unit root, the majority of transformed series ($\Delta \log(\text{Price})$) are found to be stationary at the 5\% significance level.
    \end{tablenotes}
    \end{threeparttable}
\end{table}



%%%%%%%%%%%%%%%%%%%%%%%%%%%%%%%%%%%%%%%%%%%%%%%%%%%%%%%%%%%%%%%%%%%%%%%

% Table of failed Fish et al. replication
\begin{table}[H]
    \centering
    \caption{\textcite[p. 18]{fish_algorithmic_2025} -- Table 2 replication}
    \label{tab:fe_fish}
    \begin{threeparttable}
    {\small
    \begin{tabular}{lcc}
    \toprule
    & \multicolumn{2}{c}{Dependent variable: Self Price} \\
    \cmidrule(lr){2-3}
    & (1) & (2) \\
    \midrule
    Self Price $t-1$                      & $0.9934^{***}$ & $0.9788^{***}$ \\
                                         & (0.0026)       & (0.0108)       \\
    Competitor's Price $t-1$             & $0.0029^{*}$   & $0.0081$       \\
                                         & (0.0017)       & (0.0082)       \\
    \midrule
    Model                                & P1 vs P1       & P2 vs P2       \\          
    Firm fixed effects                   & Yes            & Yes            \\
    \midrule
    Observations                         & 2,100          & 2,100          \\
    R-squared                           & 0.998          & 0.988          \\
    \bottomrule
    \end{tabular}
    }
    \begin{tablenotes}[flushleft]
    \footnotesize
    \item \textbf{Notes}: Robust standard errors in parentheses. $^{*}$ p$<$0.1, $^{**}$ p$<$0.05, $^{***}$ p$<$0.01. Models (1) and (2) examine P1 and P2's pricing responses, respectively. The high self-price coefficients (near 1.0) indicate strong price stickiness. P1 agents show marginally significant reward-punishment dynamics in response to competitor pricing, while P2 agents show no significant response to competitor moves.
    \end{tablenotes}
    \end{threeparttable}
\end{table}

%%%%%%%%%%%%%%%%%%%%%%%%%%%%%%%%%%%%%%%%%%%%%%%%%%%%%%%%%%%%%%%%%%%%%%%

\subsection*{Oligopolies}

This section presents the main empirical results, demonstrating a systematic breakdown of collusion as market concentration decreases. The analysis covers markets with 2 to 5 competing agents across different prompt specifications, providing direct evidence for Folk Theorem predictions in algorithmic settings.

Results visually demonstrate a systematic price decline with increasing market participants, consistent with the Folk Theorem predictions regarding the sustainability of collusion. Across both prompt specifications (P1, P2), average prices exhibit a monotonic decrease as $n$ increases from 2 to 5 agents, with minimal overlap in 95\% confidence intervals between different market structures. The pronounced price dispersion during initial rounds converges to distinct equilibrium levels, with duopoly markets ($n=2$) sustaining significantly higher prices than markets with four or five competitors. This pattern supports the theoretical prediction that tacit coordination becomes increasingly complex as the required discount factor $\delta \geq \frac{\pi^D - \pi^C}{\pi^D}$ approaches unity with larger $n$, ultimately rendering collusive equilibria unsustainable in less concentrated markets.

% Plot of all runs
\begin{figure}[H]
    \centering
    \includesvg[width=1\linewidth]{latex/imgs/res/price_over_time_by_prompt_prefix_combined.svg}
    \caption{Average prices over 300 rounds for markets with 2, 3, 4, and 5 agents under two prompt specifications (P1, P2). Shaded areas represent 95\% confidence intervals across 21 experimental runs per condition (7 runs × 3 $\alpha$-parameters). Prices systematically decline as the number of competing agents increases, consistent with Folk Theorem predictions on collusion sustainability.}
    \label{fig:ts_prices_comb}
\end{figure}

%%%%%%%%%%%%%%%%%%%%%%%%%%%%%%%%%%%%%%%%%%%%%%%%%%%%%%%%%%%%%%%%%%%%%%%

\subsection*{Robustness checks}\label{app:robust}

We conduct extensive robustness tests to validate our core Folk Theorem findings. These include alternative time window specifications, functional form comparisons between log and level prices, tests for non-linear effects, and bootstrap validation of coefficient stability.

% Robustness Check: Different Time Windows (Log Prices)
\begin{table}[H]
    \centering
    \caption{Robustness Check: Different Time Windows (Log Prices)}
    \label{tab:robustness_time_windows}
    \begin{threeparttable}
    \begin{tabular}{lccc}
    \toprule
     & \multicolumn{3}{c}{Dependent Variable: $\ln(\text{Price})$} \\
    \cmidrule(lr){2-4}
     & Last 25 Periods & Last 75 Periods & Last 100 Periods \\
    \midrule
    Group Size & $-0.0375^{***}$ & $-0.0369^{***}$ & $-0.0366^{***}$ \\
     & $(0.0053)$ & $(0.0056)$ & $(0.0057)$ \\
    \\
    P2 Prompt & $-0.2066^{***}$ & $-0.2108^{***}$ & $-0.2122^{***}$ \\
     & $(0.0123)$ & $(0.0127)$ & $(0.0129)$ \\
    \\
    $\alpha = 3.2$ & $0.0313^{**}$ & $0.0302^{**}$ & $0.0300^{**}$ \\
     & $(0.0138)$ & $(0.0143)$ & $(0.0145)$ \\
    \\
    $\alpha = 10.0$ & $0.0188$ & $0.0157$ & $0.0132$ \\
     & $(0.0155)$ & $(0.0160)$ & $(0.0162)$ \\
    \\
    Constant & $0.6378^{***}$ & $0.6440^{***}$ & $0.6472^{***}$ \\
     & $(0.0215)$ & $(0.0221)$ & $(0.0226)$ \\
    \midrule
    Observations & 168 & 168 & 168 \\
    R-squared & 0.679 & 0.672 & 0.667 \\
    \bottomrule
    \end{tabular}
    \begin{tablenotes}[flushleft]
    \footnotesize
    \item \textbf{Notes:} Robust standard errors (HC3) in parentheses. $^{*}$ $p<0.1$, $^{**}$ $p<0.05$, $^{***}$ $p<0.01$. Each observation represents the average log price for one experimental run over the specified final periods. Group Size ranges from 2 to 5 agents. P2 Prompt is a dummy variable for the alternative prompt specification. Results demonstrate stability of Folk Theorem findings across different convergence windows.
    \end{tablenotes}
    \end{threeparttable}
\end{table}

% Robustness Check: Price Levels vs Log Prices
\begin{table}[H]
    \centering
    \caption{Robustness Check: Price Levels vs Log Transformation}
    \label{tab:robustness_price_specification}
    \begin{threeparttable}
    \begin{tabular}{lcc}
    \toprule
     & Log Prices & Level Prices \\
     & (Last 50 Periods) & (Last 50 Periods) \\
    \midrule
    Group Size & $-0.0373^{***}$ & $-0.0528^{***}$ \\
     & $(0.0054)$ & $(0.0091)$ \\
    \\
    P2 Prompt & $-0.2082^{***}$ & $-0.3330^{***}$ \\
     & $(0.0125)$ & $(0.0209)$ \\
    \\
    $\alpha = 3.2$ & $0.0303^{**}$ & $0.0555^{**}$ \\
     & $(0.0140)$ & $(0.0238)$ \\
    \\
    $\alpha = 10.0$ & $0.0166$ & $0.0308$ \\
     & $(0.0157)$ & $(0.0260)$ \\
    \\
    Constant & $0.6417^{***}$ & $1.8692^{***}$ \\
     & $(0.0218)$ & $(0.0371)$ \\
    \midrule
    Observations & 168 & 168 \\
    R-squared & 0.675 & 0.648 \\
    \bottomrule
    \end{tabular}
    \begin{tablenotes}[flushleft]
    \footnotesize
    \item \textbf{Notes:} Robust standard errors (HC3) in parentheses. $^{*}$ $p<0.1$, $^{**}$ $p<0.05$, $^{***}$ $p<0.01$. Each observation represents the average price for one experimental run over the final 50 periods. The left column uses log-transformed prices (normalized by $\alpha$), while the right column uses price levels (normalized by $\alpha$). Group Size ranges from 2 to 5 agents. Results confirm Folk Theorem predictions are robust to functional form specification.
    \end{tablenotes}
    \end{threeparttable}
\end{table}

% Robustness Check: Non-linear and Interaction Effects
\begin{table}[H]
    \centering
    \caption{Robustness Check: Non-linear and Interaction Effects}
    \label{tab:robustness_nonlinear}
    \begin{threeparttable}
    \begin{tabular}{lcc}
    \toprule
     & \multicolumn{2}{c}{Dependent Variable: $\ln(\text{Price})$} \\
    \cmidrule(lr){2-3}
     & Squared Terms & Interaction Effects \\
    \midrule
    Group Size & $-0.0412$ & $-0.0292^{***}$ \\
     & $(0.0437)$ & $(0.0095)$ \\
    \\
    Group Size$^2$ & $0.0006$ &  \\
     & $(0.0063)$ &  \\
    \\
    P2 Prompt & $-0.2082^{***}$ & $-0.1515^{***}$ \\
     & $(0.0125)$ & $(0.0388)$ \\
    \\
    Group Size $\times$ P2 &  & $-0.0162$ \\
     &  & $(0.0109)$ \\
    \\
    $\alpha = 3.2$ & $0.0303^{**}$ & $0.0303^{**}$ \\
     & $(0.0141)$ & $(0.0138)$ \\
    \\
    $\alpha = 10.0$ & $0.0166$ & $0.0166$ \\
     & $(0.0158)$ & $(0.0158)$ \\
    \\
    Constant & $0.6478^{***}$ & $0.6133^{***}$ \\
     & $(0.0709)$ & $(0.0337)$ \\
    \midrule
    Observations & 168 & 168 \\
    R-squared & 0.675 & 0.679 \\
    \bottomrule
    \end{tabular}
    \begin{tablenotes}[flushleft]
    \footnotesize
    \item \textbf{Notes:} Robust standard errors (HC3) in parentheses. $^{*}$ $p<0.1$, $^{**}$ $p<0.05$, $^{***}$ $p<0.01$. Each observation represents the average log price for one experimental run over the final 50 periods. Left column tests for non-linear Folk Theorem effects via squared terms. The right column tests for differential group size effects across prompt types. Neither squared terms nor interaction effects are statistically significant, confirming linear Folk Theorem predictions.
    \end{tablenotes}
    \end{threeparttable}
\end{table}

\begin{table}[H]
    \centering
    \caption{Bootstrap Robustness Check: Folk Theorem Coefficient Stability}
    \label{tab:bootstrap_robustness}
    \begin{threeparttable}
    \begin{tabular}{lcc}
    \toprule
     & \multicolumn{2}{c}{Group Size Coefficient} \\
    \cmidrule(lr){2-3}
     & Without Alpha Controls & With Alpha Controls \\
    \midrule
    Original OLS Estimate & $-0.0373^{***}$ & $-0.0373^{***}$ \\
    \\
    \multicolumn{3}{l}{\textbf{Bootstrap Results (n=1,000):}} \\
    \quad Bootstrap Mean & $-0.0373$ & $-0.0375$ \\
    \quad Bootstrap SE & $0.0055$ & $0.0054$ \\
    \quad 95\% Confidence Interval & $[-0.0477, -0.0263]$ & $[-0.0475, -0.0267]$ \\
    \quad Relative SE & $0.149$ & $0.145$ \\
    \\
    \bottomrule
    \end{tabular}
    \begin{tablenotes}[flushleft]
    \footnotesize
    \item \textbf{Notes:} Bootstrap resampling (n=1,000) validates the stability of our main Folk Theorem coefficient. Both specifications show that the group size effect is robust across different sample compositions. The bootstrap mean closely matches the original OLS estimate, and the 95\% confidence intervals exclude zero, confirming that algorithmic collusion systematically decreases with group size. Relative standard errors below 0.15 indicate moderate parameter stability, supporting robust inference despite the novel "converge-and-persist" coordination patterns of LLM agents. $^{***}$ p$<$0.01.
    \end{tablenotes}
    \end{threeparttable}
\end{table}