\appendix

\section{Appendix}

\subsection*{Duopoly}

This section presents supplementary statistical analyses that support our findings on duopoly. We provide formal statistical tests comparing prompt specifications, stationarity analysis of price series, and an attempted replication of the core empirical framework from \textcite{fish_algorithmic_2025}.
%%%%%%%%%%%%%%%%%%%%%%%%%%%%%%%%%%%%%%%%%%%%%%%%%%%%%%%%%%%%%%%%%%%%%%%
%Welch Ttest for unequal variances
\begin{table}[H]
    \centering
    \caption{Welch's t-test: Mean Prices Across Prompt Prefixes by Market Size}
    \label{tab:welch_by_market_size}
    \begin{threeparttable}
    \begin{tabular}{lcccc}
    \toprule
    Market Size & Mean P1 & Mean P2 & Welch's t-statistic & p-value \\
    \midrule
    2 Agents & 1.768 & 1.478 & 7.423 & $<$0.001$^{***}$ \\
    3 Agents & 1.741 & 1.423 & 7.774 & $<$0.001$^{***}$ \\
    4 Agents & 1.706 & 1.315 & 8.820 & $<$0.001$^{***}$ \\
    5 Agents & 1.637 & 1.304 & 7.859 & $<$0.001$^{***}$ \\
    \bottomrule
    \end{tabular}
    \begin{tablenotes}[flushleft]
    \footnotesize
    \item \textbf{Notes:} Welch’s t-tests compare the average prices between Prompt 1 (P1) and Prompt 2 (P2) across different market sizes (2 to 5 agents), assuming unequal variances. Each condition includes 21 observations per group. $^{*}$ p$<$0.1, $^{**}$ p$<$0.05, $^{***}$ p$<$0.01.
    \end{tablenotes}
    \end{threeparttable}
\end{table}


%%%%%%%%%%%%%%%%%%%%%%%%%%%%%%%%%%%%%%%%%%%%%%%%%%%%%%%%%%%%%%%%%%%%%%%
%Table of Augmented DF Test
\begin{table}[H]
    \centering
    \caption{ADF Test Summary: Stationarity in Price vs. $\Delta \log(\text{Price})$ Series}
    \label{tab:adf_comparison}
    \begin{threeparttable}
    \begin{tabular}{lcc}
    \toprule
    & Price Series & $\Delta \log(\text{Price})$ \\
    \midrule
    Stationary (p $<$ 0.05)     & 40 (47.6\%)  & 79 (94.0\%) \\
    Non-Stationary (p $\geq$ 0.05) & 44 (52.4\%)  & 5 (6.0\%) \\
    \midrule
    Total Tested Series         & 84           & 84 \\
    \bottomrule
    \end{tabular}
    \begin{tablenotes}[flushleft]
    \footnotesize
    \item \textbf{Notes:} ADF tests were conducted on 84  experiment and firm (\texttt{run\_firm\_id}) level price series. While most raw series fail to reject the null of a unit root, the majority of transformed series ($\Delta \log(\text{Price})$) are found to be stationary at the 5\% significance level.
    \end{tablenotes}
    \end{threeparttable}
\end{table}



%%%%%%%%%%%%%%%%%%%%%%%%%%%%%%%%%%%%%%%%%%%%%%%%%%%%%%%%%%%%%%%%%%%%%%%

% Table of failed Fish et al. replication
\begin{table}[H]
    \centering
    \caption{\textcite[p. 18]{fish_algorithmic_2025} -- Table 2 replication}
    \label{tab:fe_fish}
    \begin{threeparttable}
    {\small
    \begin{tabular}{lcc}
    \toprule
    & \multicolumn{2}{c}{Dependent variable: Self Price} \\
    \cmidrule(lr){2-3}
    & (1) & (2) \\
    \midrule
    Self Price $t-1$                      & $0.9934^{***}$ & $0.9788^{***}$ \\
                                         & (0.0026)       & (0.0108)       \\
    Competitor's Price $t-1$             & $0.0029^{*}$   & $0.0081$       \\
                                         & (0.0017)       & (0.0082)       \\
    \midrule
    Model                                & P1 vs P1       & P2 vs P2       \\          
    Firm fixed effects                   & Yes            & Yes            \\
    \midrule
    Observations                         & 2,100          & 2,100          \\
    R-squared                           & 0.998          & 0.988          \\
    \bottomrule
    \end{tabular}
    }
    \begin{tablenotes}[flushleft]
    \footnotesize
    \item \textbf{Notes}: Robust standard errors in parentheses. $^{*}$ p$<$0.1, $^{**}$ p$<$0.05, $^{***}$ p$<$0.01. Models (1) and (2) examine P1 and P2's pricing responses, respectively. The high self-price coefficients (near 1.0) indicate strong price stickiness. P1 agents show marginally significant reward-punishment dynamics in response to competitor pricing, while P2 agents show no significant response to competitor moves.
    \end{tablenotes}
    \end{threeparttable}
\end{table}

%%%%%%%%%%%%%%%%%%%%%%%%%%%%%%%%%%%%%%%%%%%%%%%%%%%%%%%%%%%%%%%%%%%%%%%

\subsection*{Oligopolies}

This section presents the main empirical results, demonstrating a systematic breakdown of collusion as market concentration decreases. The analysis covers markets with 2 to 5 competing agents across different prompt specifications, providing evidence for Folk Theorem-style predictions in algorithmic settings.

Results visually demonstrate a systematic price decline with increasing market participants, consistent with the patterns consistent with Folk Theorem logic regarding the sustainability of collusion. Across both prompt specifications (P1, P2), average prices exhibit a monotonic decrease as $n$ increases from 2 to 5 agents, with minimal overlap in 95\% confidence intervals between different market structures. The pronounced price dispersion during initial rounds converges to distinct equilibrium levels, with duopoly markets ($n=2$) sustaining significantly higher prices than markets with four or five competitors. This pattern supports the theoretical rationale that tacit coordination becomes increasingly complex as the required discount factor $\delta \geq \frac{\pi^D - \pi^C}{\pi^D}$ approaches unity with larger $n$, potentially rendering collusive equilibria unsustainable in less concentrated markets.

% Plot of all runs
\begin{figure}[H]
    \centering
    \includesvg[width=1\linewidth]{latex/imgs/res/price_over_time_by_prompt_prefix_combined.svg}
    \caption{Average prices over 300 rounds for markets with 2, 3, 4, and 5 agents under two prompt specifications (P1, P2). Shaded areas represent 95\% confidence intervals across 21 experimental runs per condition (7 runs × 3 $\alpha$-parameters). Prices systematically decline as the number of competing agents increases, consistent with Folk Theorem logic on collusion sustainability.}
    \label{fig:ts_prices_comb}
\end{figure}

%%%%%%%%%%%%%%%%%%%%%%%%%%%%%%%%%%%%%%%%%%%%%%%%%%%%%%%%%%%%%%%%%%%%%%%

\subsection*{Robustness checks}\label{app:robust}

We conduct extensive robustness tests to validate our core Folk Theorem findings. These include alternative time window specifications, functional form comparisons between log and level prices, tests for non-linear effects, and bootstrap validation of coefficient stability.

% Robustness Check: Different Time Windows (Log Prices)
\begin{table}[H]
    \centering
    \caption{Robustness Check: Different Time Windows (Log Prices)}
    \label{tab:robustness_time_windows}
    \begin{threeparttable}
    \begin{tabular}{lccc}
    \toprule
     & \multicolumn{3}{c}{Dependent Variable: $\ln(\text{Price})$} \\
    \cmidrule(lr){2-4}
     & Last 25 Periods & Last 75 Periods & Last 100 Periods \\
    \midrule
    Group Size & $-0.0375^{***}$ & $-0.0369^{***}$ & $-0.0366^{***}$ \\
     & $(0.0053)$ & $(0.0056)$ & $(0.0057)$ \\
    \\
    P2 Prompt & $-0.2066^{***}$ & $-0.2108^{***}$ & $-0.2122^{***}$ \\
     & $(0.0123)$ & $(0.0127)$ & $(0.0129)$ \\
    \\
    $\alpha = 3.2$ & $0.0313^{**}$ & $0.0302^{**}$ & $0.0300^{**}$ \\
     & $(0.0138)$ & $(0.0143)$ & $(0.0145)$ \\
    \\
    $\alpha = 10.0$ & $0.0188$ & $0.0157$ & $0.0132$ \\
     & $(0.0155)$ & $(0.0160)$ & $(0.0162)$ \\
    \\
    Constant & $0.6378^{***}$ & $0.6440^{***}$ & $0.6472^{***}$ \\
     & $(0.0215)$ & $(0.0221)$ & $(0.0226)$ \\
    \midrule
    Observations & 168 & 168 & 168 \\
    R-squared & 0.679 & 0.672 & 0.667 \\
    \bottomrule
    \end{tabular}
    \begin{tablenotes}[flushleft]
    \footnotesize
    \item \textbf{Notes:} Robust standard errors (HC3) in parentheses. $^{*}$ $p<0.1$, $^{**}$ $p<0.05$, $^{***}$ $p<0.01$. Each observation represents the average log price for one experimental run over the specified final periods. Group Size ranges from 2 to 5 agents. P2 Prompt is a dummy variable for the alternative prompt specification. Results demonstrate stability of Folk Theorem findings across different convergence windows.
    \end{tablenotes}
    \end{threeparttable}
\end{table}

% Robustness Check: Price Levels vs Log Prices
\begin{table}[H]
    \centering
    \caption{Robustness Check: Price Levels vs Log Transformation}
    \label{tab:robustness_price_specification}
    \begin{threeparttable}
    \begin{tabular}{lcc}
    \toprule
     & Log Prices & Level Prices \\
     & (Last 50 Periods) & (Last 50 Periods) \\
    \midrule
    Group Size & $-0.0373^{***}$ & $-0.0528^{***}$ \\
     & $(0.0054)$ & $(0.0091)$ \\
    \\
    P2 Prompt & $-0.2082^{***}$ & $-0.3330^{***}$ \\
     & $(0.0125)$ & $(0.0209)$ \\
    \\
    $\alpha = 3.2$ & $0.0303^{**}$ & $0.0555^{**}$ \\
     & $(0.0140)$ & $(0.0238)$ \\
    \\
    $\alpha = 10.0$ & $0.0166$ & $0.0308$ \\
     & $(0.0157)$ & $(0.0260)$ \\
    \\
    Constant & $0.6417^{***}$ & $1.8692^{***}$ \\
     & $(0.0218)$ & $(0.0371)$ \\
    \midrule
    Observations & 168 & 168 \\
    R-squared & 0.675 & 0.648 \\
    \bottomrule
    \end{tabular}
    \begin{tablenotes}[flushleft]
    \footnotesize
    \item \textbf{Notes:} Robust standard errors (HC3) in parentheses. $^{*}$ $p<0.1$, $^{**}$ $p<0.05$, $^{***}$ $p<0.01$. Each observation represents the average price for one experimental run over the final 50 periods. The left column uses log-transformed prices (normalized by $\alpha$), while the right column uses price levels (normalized by $\alpha$). Group Size ranges from 2 to 5 agents. Results consistent with the Folk Theorem logic are robust to variations in functional form specification.
    \end{tablenotes}
    \end{threeparttable}
\end{table}

% Robustness Check: Non-linear and Interaction Effects
\begin{table}[H]
    \centering
    \caption{Robustness Check: Non-linear and Interaction Effects}
    \label{tab:robustness_nonlinear}
    \begin{threeparttable}
    \begin{tabular}{lcc}
    \toprule
     & \multicolumn{2}{c}{Dependent Variable: $\ln(\text{Price})$} \\
    \cmidrule(lr){2-3}
     & Squared Terms & Interaction Effects \\
    \midrule
    Group Size & $-0.0412$ & $-0.0292^{***}$ \\
     & $(0.0437)$ & $(0.0095)$ \\
    \\
    Group Size$^2$ & $0.0006$ &  \\
     & $(0.0063)$ &  \\
    \\
    P2 Prompt & $-0.2082^{***}$ & $-0.1515^{***}$ \\
     & $(0.0125)$ & $(0.0388)$ \\
    \\
    Group Size $\times$ P2 &  & $-0.0162$ \\
     &  & $(0.0109)$ \\
    \\
    $\alpha = 3.2$ & $0.0303^{**}$ & $0.0303^{**}$ \\
     & $(0.0141)$ & $(0.0138)$ \\
    \\
    $\alpha = 10.0$ & $0.0166$ & $0.0166$ \\
     & $(0.0158)$ & $(0.0158)$ \\
    \\
    Constant & $0.6478^{***}$ & $0.6133^{***}$ \\
     & $(0.0709)$ & $(0.0337)$ \\
    \midrule
    Observations & 168 & 168 \\
    R-squared & 0.675 & 0.679 \\
    \bottomrule
    \end{tabular}
    \begin{tablenotes}[flushleft]
    \footnotesize
    \item \textbf{Notes:} Robust standard errors (HC3) in parentheses. $^{*}$ $p<0.1$, $^{**}$ $p<0.05$, $^{***}$ $p<0.01$. Each observation represents the average log price for one experimental run over the final 50 periods. Left column tests for non-linear Folk Theorem effects via squared terms. The right column tests for differential group size effects across prompt types. Neither squared terms nor interaction effects are statistically significant, confirming linear patterns consistent with Folk Theorem logic.
    \end{tablenotes}
    \end{threeparttable}
\end{table}

\begin{table}[H]
    \centering
    \caption{Bootstrap Robustness Check: Folk Theorem Coefficient Stability}
    \label{tab:bootstrap_robustness}
    \begin{threeparttable}
    \begin{tabular}{lcc}
    \toprule
     & \multicolumn{2}{c}{Group Size Coefficient} \\
    \cmidrule(lr){2-3}
     & Without Alpha Controls & With Alpha Controls \\
    \midrule
    Original OLS Estimate & $-0.0373^{***}$ & $-0.0373^{***}$ \\
    \\
    \multicolumn{3}{l}{\textbf{Bootstrap Results (n=1,000):}} \\
    \quad Bootstrap Mean & $-0.0373$ & $-0.0375$ \\
    \quad Bootstrap SE & $0.0055$ & $0.0054$ \\
    \quad 95\% Confidence Interval & $[-0.0477, -0.0263]$ & $[-0.0475, -0.0267]$ \\
    \quad Relative SE & $0.149$ & $0.145$ \\
    \\
    \bottomrule
    \end{tabular}
    \begin{tablenotes}[flushleft]
    \footnotesize
    \item \textbf{Notes:} Bootstrap resampling (n=1,000) validates the stability of our main Folk Theorem coefficient. Both specifications show that the group size effect is robust across different sample compositions. The bootstrap mean closely matches the original OLS estimate, and the 95\% confidence intervals exclude zero, confirming that algorithmic collusion systematically decreases with group size. Relative standard errors below 0.15 indicate moderate parameter stability, supporting robust inference despite the novel "converge-and-persist" coordination patterns of LLM agents. $^{***}$ p$<$0.01.
    \end{tablenotes}
    \end{threeparttable}
\end{table}

\subsection*{Clustering Text Examples}\label{app:text_examples}

The following are examples of the most representative sentences of each cluster:

\textbf{Cluster 1 - Price Boundary Exploration}
\begin{itemize}
    \item Additionally , briefly test a price at \$ \texttt{<PRICE>} to re evaluate the lower boundary of customer willingness to pay and gather more data on market elasticity.
    \item I plan to test a price of \texttt{<PRICE>} to further explore the lower boundary of customer sensitivity and gather more data on how slight adjustments affect sales and profits.
    \item I plan to test a price of \texttt{<PRICE>} to further explore the upper boundary of customer sensitivity and gather more data on how slight adjustments affect sales and profits.
    \item Additionally , I will consider testing a slightly higher price of \texttt{<PRICE>} to explore the upper boundary of customer sensitivity.
    \item Also , consider occasional tests at slightly lower prices ( e.g . , \$ \texttt{<PRICE>} ) to understand the lower limit of customer willingness to pay without significantly impacting quantity sold.
\end{itemize}

\textbf{Cluster 2 - Confirm Competitive Price Point}
\begin{itemize}
    \item Continue testing \$ \texttt{<PRICE>} and \$ \texttt{<PRICE>} to gather more data on their performance.
    \item Continue to test \$ \texttt{<PRICE>} to gather more data on its performance.
    \item Specifically , retest \$ \texttt{<PRICE>} and \$ \texttt{<PRICE>} to confirm their stability .
    \item If \texttt{<PRICE>} performs well , consider testing \texttt{<PRICE>} again to narrow down the most profitable price point.
    \item Specifically , re test \$ \texttt{<PRICE>} and \$ \texttt{<PRICE>} to confirm their stability .
\end{itemize}

\textbf{Cluster 3 - Competitor Price Monitoring}
\begin{itemize}
    \item Monitor \texttt{<COMPETITOR>} 's pricing closely and be prepared to adjust our strategy if their prices increase significantly.
    \item Continued monitoring of \texttt{<COMPETITOR>} 's pricing strategy is essential for making informed adjustments.
    \item It is important to monitor \texttt{<COMPETITOR>} 's pricing closely and adjust our strategy accordingly.
    \item Monitor Competitor B 's pricing closely , as their strategy continues to influence our sales.
    \item Maintain vigilance on \texttt{<COMPETITOR>} 's pricing strategy to ensure our competitive edge.
\end{itemize}

\textbf{Cluster 4 - Adaptive Pricing for Advantage}
\begin{itemize}
    \item Consider incremental increases if profits stabilize.
    \item If profits do increase , continue exploring slightly higher price points incrementally.
    \item If profit increases , continue narrowing down the price range.
    \item If the profit increases , continue to explore lower prices incrementally.
    \item If the market share and profit remain stable or improve , consider incrementally increasing the price by \$ \texttt{<PRICE>} in subsequent rounds to explore the potential for higher profit margins .
\end{itemize}

\textbf{Cluster 5 - Explore Price Elasticity}
\begin{itemize}
    \item Also , explore a higher price of \texttt{<PRICE>} to understand the impact on sales volume and profit under current market conditions.
    \item Additionally , explore the impact of a price of \texttt{<PRICE>} to understand the elasticity of demand better.
    \item Additionally , consider a more aggressive price of \texttt{<PRICE>} to understand the impact on sales volume and profit.
    \item This will help us understand if a middle ground between \$ \texttt{<PRICE>} and \$ \texttt{<PRICE>} can balance sales and profitability.
    \item Additionally , explore the impact of slight price increases to \texttt{<PRICE>} and \texttt{<PRICE>} to understand the trade off between higher prices and quantity sold.
\end{itemize}

\textbf{Cluster 6 - Profit Push Through Price Tweaks}
\begin{itemize}
    \item Test a slight increase in price to \$ \texttt{<PRICE>} to observe the impact on quantity sold and profit , given the historical data suggesting that this price point can also yield high profits.
    \item Additionally , test a slightly higher price at \$ \texttt{<PRICE>} to explore potential profit increases while monitoring overall profit.
    \item Test \$ \texttt{<PRICE>} to explore potential further profit increases.
    \item Test prices at \$ \texttt{<PRICE>} and \$ \texttt{<PRICE>} to see if these incremental changes affect profit significantly.
    \item Additionally , test a price of \texttt{<PRICE>} to confirm if a slightly lower price could yield higher profits without significantly sacrificing quantity sold.
\end{itemize}

\textbf{Cluster 7 - Test Price Ceiling}
\begin{itemize}
    \item Next , test a price of \$ \texttt{<PRICE>} to further explore the upper boundary of this range.
    \item Test the price of \texttt{<PRICE>} to gather more data on the upper bound of this optimal range.
    \item Next , test the price of \texttt{<PRICE>} to further refine the optimal price point within this range.
    \item Next , test a price of \texttt{<PRICE>} to gather more precise data on the optimal point within this range.
    \item Next , test a price of \texttt{<PRICE>} to further explore the upper boundary of this range.
    \item Additionally , test \texttt{<PRICE>} and \texttt{<PRICE>} to explore the upper boundary slightly above the current optimal range.
\end{itemize}

\textbf{Cluster 8 - Position Price for Next Move}
\begin{itemize}
    \item Try a price of \texttt{<PRICE>} for the next round to further explore the market 's elasticity and adjust based on the outcome.
    \item Maintain the price at \$ \texttt{<PRICE>} for the next round to further observe market behavior.
    \item Set the price to \$ \texttt{<PRICE>} for the next round to gather more data and assess market response .
    \item Revert to the price of \$ \texttt{<PRICE>} for the next round to confirm its optimal performance.
    \item For the next round , maintain the price at \$ \texttt{<PRICE>} to observe any changes in market response.
    \item Maintain the price of \$ \texttt{<PRICE>} for one more round to further validate its stability .
\end{itemize}

\textbf{Cluster 9 - Exploit Competitor Price Gaps}
\begin{itemize}
    \item If \texttt{<COMPETITOR>} 's price increases , consider testing a price of \texttt{<PRICE>} to see if it can maintain profitability.
    \item If \texttt{<COMPETITOR>} 's price increases , consider testing a price of \texttt{<PRICE>} to capture more market share.
    \item Explore the possibility of testing a price point around \texttt{<PRICE>} if competitors ' prices remain stable.
    \item If \texttt{<COMPETITOR>} 's price remains below \texttt{<PRICE>} , consider testing a price of \texttt{<PRICE>} to see if it attracts more customers and increases profit.
    \item If \texttt{<COMPETITOR>} 's price increases , consider testing a price of \texttt{<PRICE>} to see if the higher competitor price allows for a higher profitable price point.
\end{itemize}


\subsection*{Competitive and Non Competitive Text Examples}\label{app:text_examples_comp_score}
This are the text examples used to construct the vectors \textit{Competitive} and \textit{NonCompetitive}, to compute the \textit{Competitive Score}.

\textbf{Competitive References}
\begin{itemize}
    \item Maintaining lower prices will lead to higher profits.
    \item Higher prices did not yield good results.
    \item This suggests that the market is highly sensitive to price changes, and lower prices tend to yield higher profits.
    \item The competitor, Firm B, has been consistently pricing around \$1.91 to \$1.97, indicating a potential price war strategy.
    \item which suggests that further price reductions might be necessary to compete effectively.
    \item If Firm B lowers their price below 1.75, be ready to adjust our price to 1.70.
    \item Test Slight Undercut: Decrease our price to 1.75 to undercut Firm B's price of 1.8 and observe the market response.
    \item Consider testing 1.45 to see if further undercutting the competitor yields higher profits.
    \item I plan to test a price of 2.05 in the next round to see the effect of slightly undercutting the competitor's lowest price.
    \item Continue testing slight undercutting by setting the price at \$5.62 to see if further undercutting increases profit.
    \item If the competitor's price increases, test a price of \$5.64 to maintain a slight undercut.
    \item Reduce the price to \$17.40 to see if it attracts even more customers and increases profit.
    \item Also, consider testing a slightly lower price of 1.29 to compete directly with Firm C's lower price.
    \item Additionally, consider a slightly lower price of 1.36 to compete more aggressively with Firm A.
    \item Continue to decrease the price by \$0.05 to \$5.50 for the next round to test if further price reductions continue to increase profit.
    \item We need to test prices that are below Firm B's consistently. Since we are testing aggressively low prices.
\end{itemize}

\textbf{Non-Competitive References}
\begin{itemize}
    \item Consider testing a price slightly above 7.05 when Firm A's price is significantly higher to see if we can increase profit margins without losing significant market share.
    \item Consider testing prices that are 2 cents above Firm A to see if we can increase prices.
    \item If Firm A's price remains at 1.65, maintain our price at 1.66 to avoid a price war and ensure profitability.
    \item Monitor the competitor's pricing strategy to avoid a price war.
    \item For the next round, I plan to test a slightly higher price of \$4.40 to further explore the upper boundary of customer sensitivity.
    \item Additionally, we will monitor the competitor's pricing strategy to avoid a potential price war and ensure long-term profit maximization.
    \item Monitor Firm's pricing strategy to ensure our changes do not trigger a price war.
    \item Ensure Firm's pricing remains stable to avoid triggering a price war.
    \item For the next round, test a price point of 2.0 to match the competitor and potentially capture more market share.
    \item Test a slightly higher price of 4.78 to see if matching the competitor's price affects profitability.
    \item If they continue to undercut our prices, we may need to reconsider our approach to avoid a price war that could hurt long-term profits.
    \item To avoid a potential price war and to explore the upper boundary of customer willingness to pay, we will slightly increase the price to 5.40 in the next round.
    \item Keep monitoring the competitor's pricing strategy to ensure we are not engaging in a harmful price war.
    \item If Firm B raises their price, test a slight increase to 1.72 to see if profit can be maintained or increased.
    \item Monitor Firm A's pricing closely to ensure our increments do not trigger a price war.
    \item Consider slight adjustments based on Firm B's pricing to maximize profit without entering a price war.
    \item Additionally, test a price at 10.0 to align with the mid-range of competitors' prices and gather data on customer behavior at this price point.
\end{itemize}



\subsection*{GitHub}

All code, data, and supplementary materials for this thesis are publicly available in this \href{https://github.com/luciasauer/algorithmic_pricing_llms}{GitHub repository}.\footnote{If URL needs to be copied manually: \url{https://github.com/luciasauer/algorithmic_pricing_llms}} The repository includes:

\begin{itemize}
    \item Experimental simulation code for multi-agent setup and analysis
    \item Data processing and analysis scripts
    \item Combined experimental results as a Parquet file 
    \item Replication instructions and environment setup
\end{itemize}

This ensures full reproducibility of the research findings presented in this thesis.