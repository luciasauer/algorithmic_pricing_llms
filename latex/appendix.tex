\appendix

\section{Appendix}

\subsection*{Duopoloy}

\begin{table}[H]
    \centering
    \caption{Welch's t-test: Average Prices Across Prompt Prefixes}
    \label{tab:welch_test_1}
    \begin{threeparttable}
    \begin{tabular}{lcc}
    \toprule
    & P1 & P2 \\
    \midrule
    Mean Price (last 50 rounds) &  &  \\
    Standard Deviation          &  &  \\
    Observations                &  &  \\
    \midrule
    Difference in Means         & \multicolumn{2}{c}{$^{***}$} \\
    Welch's t-statistic         & \multicolumn{2}{c}{} \\
    Degrees of Freedom          & \multicolumn{2}{c}{} \\
    p-value                     & \multicolumn{2}{c}{<} \\
    \bottomrule
    \end{tabular}
    \begin{tablenotes}[flushleft]
    \footnotesize
    \item \textbf{Notes:} Welch's t-test compares the mean prices set under each prompt prefix across the 21 experiments, assuming unequal variances. $^{*}$ p$<$0.1, $^{**}$ p$<$0.05, $^{***}$ p$<$0.01. Mean prices are calculated using the final 50 rounds of each experiment.
    \end{tablenotes}
    \end{threeparttable}
\end{table}

%%%%%%%%%%%%%%%%%%%%%%%%%%%%%%%%%%%%%%%%%%%%%%%%%%%%%%%%%%%%%%%%%%%%%%%

% Table of failed Fish et al. replication
\begin{table}[H]
    \centering
    \caption{\textcite[p. 18]{fish_algorithmic_2025} -- Table 2 replication}
    \label{tab:fe_fish}
    \begin{threeparttable}
    {\small
    \begin{tabular}{lcc}
    \toprule
    & \multicolumn{2}{c}{Dependent variable: Self Price} \\
    \cmidrule(lr){2-3}
    & (1) & (2) \\
    \midrule
    Self Price $t-1$                      & $0.9934^{***}$ & $0.9788^{***}$ \\
                                         & (0.0026)       & (0.0108)       \\
    Competitor's Price $t-1$             & $0.0029^{*}$   & $0.0081$       \\
                                         & (0.0017)       & (0.0082)       \\
    \midrule
    Model                                & P1 vs P1       & P2 vs P2       \\          
    Firm fixed effects                   & Yes            & Yes            \\
    \midrule
    Observations                         & 2,100          & 2,100          \\
    R-squared                           & 0.998          & 0.988          \\
    \bottomrule
    \end{tabular}
    }
    \begin{tablenotes}[flushleft]
    \footnotesize
    \item \textbf{Notes}: Robust standard errors in parentheses. $^{*}$ p$<$0.1, $^{**}$ p$<$0.05, $^{***}$ p$<$0.01. Models (1) and (2) examine P1 and P2's pricing responses, respectively. The high self-price coefficients (near 1.0) indicate strong price stickiness. P1 agents show marginally significant reward-punishment dynamics in response to competitor pricing, while P2 agents show no significant response to competitor moves.
    \end{tablenotes}
    \end{threeparttable}
\end{table}

%%%%%%%%%%%%%%%%%%%%%%%%%%%%%%%%%%%%%%%%%%%%%%%%%%%%%%%%%%%%%%%%%%%%%%%

\subsection*{Oligopolies}

% Plot of all runs
\begin{figure}[H]
    \centering
    \includesvg[width=1\linewidth]{latex/imgs/res/price_over_time_by_prompt_prefix_combined.svg}
    \caption{Average Price by Prompt and Experiment, shaded areas are 95\% confidence intervals, lines are averaged values out of all 21 runs per prompt (7 runs $\times$ 3 alphas).}
    \label{fig:ts_prices_comb}
\end{figure}

%%%%%%%%%%%%%%%%%%%%%%%%%%%%%%%%%%%%%%%%%%%%%%%%%%%%%%%%%%%%%%%%%%%%%%%

\subsection*{Robustness checks}\label{app:robust}

% Robustness Check: Different Time Windows (Log Prices)
\begin{table}[H]
    \centering
    \caption{Robustness Check: Different Time Windows (Log Prices)}
    \label{tab:robustness_time_windows}
    \begin{threeparttable}
    \begin{tabular}{lccc}
    \toprule
     & \multicolumn{3}{c}{Dependent Variable: $\ln(\text{Price})$} \\
    \cmidrule(lr){2-4}
     & Last 25 Periods & Last 75 Periods & Last 100 Periods \\
    \midrule
    Group Size & $-0.0375^{***}$ & $-0.0369^{***}$ & $-0.0366^{***}$ \\
     & $(0.0053)$ & $(0.0056)$ & $(0.0057)$ \\
    \\
    P2 Prompt & $-0.2066^{***}$ & $-0.2108^{***}$ & $-0.2122^{***}$ \\
     & $(0.0123)$ & $(0.0127)$ & $(0.0129)$ \\
    \\
    $\alpha = 3.2$ & $0.0313^{**}$ & $0.0302^{**}$ & $0.0300^{**}$ \\
     & $(0.0138)$ & $(0.0143)$ & $(0.0145)$ \\
    \\
    $\alpha = 10.0$ & $0.0188$ & $0.0157$ & $0.0132$ \\
     & $(0.0155)$ & $(0.0160)$ & $(0.0162)$ \\
    \\
    Constant & $0.6378^{***}$ & $0.6440^{***}$ & $0.6472^{***}$ \\
     & $(0.0215)$ & $(0.0221)$ & $(0.0226)$ \\
    \midrule
    Observations & 168 & 168 & 168 \\
    R-squared & 0.679 & 0.672 & 0.667 \\
    \bottomrule
    \end{tabular}
    \begin{tablenotes}[flushleft]
    \footnotesize
    \item \textbf{Notes:} Robust standard errors (HC3) in parentheses. $^{*}$ $p<0.1$, $^{**}$ $p<0.05$, $^{***}$ $p<0.01$. Each observation represents the average log price for one experimental run over the specified final periods. Group Size ranges from 2 to 5 agents. P2 Prompt is a dummy variable for the alternative prompt specification. Results demonstrate stability of Folk Theorem findings across different convergence windows.
    \end{tablenotes}
    \end{threeparttable}
\end{table}

% Robustness Check: Price Levels vs Log Prices
\begin{table}[H]
    \centering
    \caption{Robustness Check: Price Levels vs Log Transformation}
    \label{tab:robustness_price_specification}
    \begin{threeparttable}
    \begin{tabular}{lcc}
    \toprule
     & Log Prices & Level Prices \\
     & (Last 50 Periods) & (Last 50 Periods) \\
    \midrule
    Group Size & $-0.0373^{***}$ & $-0.0528^{***}$ \\
     & $(0.0054)$ & $(0.0091)$ \\
    \\
    P2 Prompt & $-0.2082^{***}$ & $-0.3330^{***}$ \\
     & $(0.0125)$ & $(0.0209)$ \\
    \\
    $\alpha = 3.2$ & $0.0303^{**}$ & $0.0555^{**}$ \\
     & $(0.0140)$ & $(0.0238)$ \\
    \\
    $\alpha = 10.0$ & $0.0166$ & $0.0308$ \\
     & $(0.0157)$ & $(0.0260)$ \\
    \\
    Constant & $0.6417^{***}$ & $1.8692^{***}$ \\
     & $(0.0218)$ & $(0.0371)$ \\
    \midrule
    Observations & 168 & 168 \\
    R-squared & 0.675 & 0.648 \\
    \bottomrule
    \end{tabular}
    \begin{tablenotes}[flushleft]
    \footnotesize
    \item \textbf{Notes:} Robust standard errors (HC3) in parentheses. $^{*}$ $p<0.1$, $^{**}$ $p<0.05$, $^{***}$ $p<0.01$. Each observation represents the average price for one experimental run over the final 50 periods. The left column uses log-transformed prices (normalized by $\alpha$), while the right column uses price levels (normalized by $\alpha$). Group Size ranges from 2 to 5 agents. Results confirm Folk Theorem predictions are robust to functional form specification.
    \end{tablenotes}
    \end{threeparttable}
\end{table}

% Robustness Check: Non-linear and Interaction Effects
\begin{table}[H]
    \centering
    \caption{Robustness Check: Non-linear and Interaction Effects}
    \label{tab:robustness_nonlinear}
    \begin{threeparttable}
    \begin{tabular}{lcc}
    \toprule
     & \multicolumn{2}{c}{Dependent Variable: $\ln(\text{Price})$} \\
    \cmidrule(lr){2-3}
     & Squared Terms & Interaction Effects \\
    \midrule
    Group Size & $-0.0412$ & $-0.0292^{***}$ \\
     & $(0.0437)$ & $(0.0095)$ \\
    \\
    Group Size$^2$ & $0.0006$ &  \\
     & $(0.0063)$ &  \\
    \\
    P2 Prompt & $-0.2082^{***}$ & $-0.1515^{***}$ \\
     & $(0.0125)$ & $(0.0388)$ \\
    \\
    Group Size $\times$ P2 &  & $-0.0162$ \\
     &  & $(0.0109)$ \\
    \\
    $\alpha = 3.2$ & $0.0303^{**}$ & $0.0303^{**}$ \\
     & $(0.0141)$ & $(0.0138)$ \\
    \\
    $\alpha = 10.0$ & $0.0166$ & $0.0166$ \\
     & $(0.0158)$ & $(0.0158)$ \\
    \\
    Constant & $0.6478^{***}$ & $0.6133^{***}$ \\
     & $(0.0709)$ & $(0.0337)$ \\
    \midrule
    Observations & 168 & 168 \\
    R-squared & 0.675 & 0.679 \\
    \bottomrule
    \end{tabular}
    \begin{tablenotes}[flushleft]
    \footnotesize
    \item \textbf{Notes:} Robust standard errors (HC3) in parentheses. $^{*}$ $p<0.1$, $^{**}$ $p<0.05$, $^{***}$ $p<0.01$. Each observation represents the average log price for one experimental run over the final 50 periods. Left column tests for non-linear Folk Theorem effects via squared terms. The right column tests for differential group size effects across prompt types. Neither squared terms nor interaction effects are statistically significant, confirming linear Folk Theorem predictions.
    \end{tablenotes}
    \end{threeparttable}
\end{table}

\begin{table}[H]
    \centering
    \caption{Bootstrap Robustness Check: Folk Theorem Coefficient Stability}
    \label{tab:bootstrap_robustness}
    \begin{threeparttable}
    \begin{tabular}{lcc}
    \toprule
     & \multicolumn{2}{c}{Group Size Coefficient} \\
    \cmidrule(lr){2-3}
     & Without Alpha Controls & With Alpha Controls \\
    \midrule
    Original OLS Estimate & $-0.0373^{***}$ & $-0.0373^{***}$ \\
    \\
    \multicolumn{3}{l}{\textbf{Bootstrap Results (n=1,000):}} \\
    \quad Bootstrap Mean & $-0.0373$ & $-0.0375$ \\
    \quad Bootstrap SE & $0.0055$ & $0.0054$ \\
    \quad 95\% Confidence Interval & $[-0.0477, -0.0263]$ & $[-0.0475, -0.0267]$ \\
    \quad Relative SE & $0.149$ & $0.145$ \\
    \\
    \bottomrule
    \end{tabular}
    \begin{tablenotes}[flushleft]
    \footnotesize
    \item \textbf{Notes:} Bootstrap resampling (n=1,000) validates the stability of our main Folk Theorem coefficient. Both specifications show that the group size effect is robust across different sample compositions. The bootstrap mean closely matches the original OLS estimate, and the 95\% confidence intervals exclude zero, confirming that algorithmic collusion systematically decreases with group size. Relative standard errors below 0.15 indicate moderate parameter stability, supporting robust inference despite the novel "converge-and-persist" coordination patterns of LLM agents. $^{***}$ p$<$0.01.
    \end{tablenotes}
    \end{threeparttable}
\end{table}