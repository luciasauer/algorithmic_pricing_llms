% Here starts the preamble
\documentclass[12pt]{article}
\input{latex/packages}
\input{latex/commands}

\begin{document}

% First Cover (for presentation or preliminary cover)
\pdfbookmark[1]{Preliminary Cover}{Preliminary Cover}
\begin{titlepage}
    \centering
    \includesvg[width=0.75\textwidth]{latex/imgs/BSE Barcelona Graduate School of Economics.svg}\par
    \vspace{1.5cm}

    {\LARGE\bfseries Master's Degree in Data Science\par}
    \vspace{0.3cm}
    {\LARGE Data Science for Decision Making Programme\par}

    \vfill

    {\huge\bfseries \en{Folk Around and Find Out: Algorithmic Collusion and the Limits of Coordination}\par}

    \vfill

    {\Large Authors:\par}
    {\Large Moritz Peist, Julián Romero, Lucia Sauer\par}
    \vspace{0.5cm}
    {\Large Supervisors:\par}
    {\Large Christopher Rauh, Hannes Mueller\par}

    \vfill

    {\Large July 2025\par}
\end{titlepage}

% Mandatory for abstract
\pdfbookmark[1]{Abstract}{abstract}
\begin{titlepage}
    % Abstract title styling
    {\LARGE\bfseries Abstract in English\par}
    \vspace{0.5cm}
    The Folk Theorem establishes that collusion can be sustained in repeated interactions, yet empirical evidence suggests coordination becomes more difficult as market participants increase. This thesis presents the first test of whether Large Language Model (LLM) agents exhibit this pattern. In controlled experiments with 2--5 competing agents, we find LLM coordination erodes predictably with competition. Our results show a 3.7\% reduction in equilibrium price for each additional firm (p < 0.001), with prices declining smoothly. This culminates in a 10.6\% total price reduction from duopoly to five-agent markets, providing quantitative evidence on algorithmic collusion boundaries in the AI era.

    \vspace{1.5cm}

    {\LARGE\bfseries Abstract in Spanish\par}
    \vspace{0.5cm}
    El Teorema de Folk establece que la colusión puede mantenerse en interacciones repetidas, pero la evidencia empírica sugiere que la coordinación se vuelve más difícil a más participantes en el mercado. Esta tesis presenta una primera prueba  de si agentes de modelos de lenguaje grandes (LLM) muestran este patrón. En experimentos controlados con 2-5 agentes competidores, encontramos que la coordinación entre LLM se erosiona predeciblemente con la competencia. Resultados muestran una reducción del 3,7\% en el precio de equilibrio por empresa adicional (p < 0,001), con disminución precios. Esto culmina en reducción total del 10,6\% desde duopolio hasta mercados con cinco agentes, proporcionando evidencia cuantitativa sobre límites colusión algorítmica en era de la IA. 

    \vfill

    {\large \textbf{Keywords in English}: algorithmic collusion, Folk Theorem, LLM agents.\par}
    
    \vspace{0.5cm}

    {\large \textbf{Keywords in Spanish}: colusión algorítmica, Teorema de Folk, agentes LLM.\par}
    \vspace{2cm}

\end{titlepage}

\pdfbookmark[1]{Title page}{Title page}
\begin{titlepage}
	\centering
	\includesvg[width=0.75\textwidth]{latex/imgs/BSE Barcelona Graduate School of Economics.svg}\par\vspace{1cm}
	{\huge\bfseries Folk Around and Find Out:\par}
    {\large\bfseries Algorithmic Collusion and the Limits of Coordination\par}
	\vspace{1cm}
    \noindent\rule{\textwidth}{1pt}
    {\Large Moritz Peist (254017)\par}
    {\Large Julián Romero (253764)\par}
    {\Large Lucia Sauer (254053)\par}
    \noindent\rule{\textwidth}{1pt}
    \vfill
    \begin{abstract}
        \noindent
        The \emph{Folk Theorem} establishes that collusion can be sustained in repeated interactions, yet empirical evidence suggests coordination becomes more difficult as market participants increase. This thesis presents the first test of whether Large Language Model (LLM) agents exhibit this pattern. In controlled experiments with 2-5 competing agents, we find LLM coordination erodes predictably with competition. Our results show a 3.7\% reduction in equilibrium price for each additional firm (p < 0.001), with prices declining smoothly. This culminates in a 10.6\% total price reduction from duopoly to five-agent markets, providing quantitative evidence on algorithmic collusion boundaries in the AI era.
    \end{abstract}
	\vfill
    % Bottom of the page
	{\large \today\par}
\end{titlepage}

\tableofcontents
\thispagestyle{empty}

\newpage
\addtocounter{page}{-1}
\include*{latex/chapters/part1}
\include*{latex/chapters/part2}
\include*{latex/chapters/part3}
\include*{latex/chapters/part4}
\include*{latex/chapters/part5}
\include*{latex/chapters/part6}

\newpage
\printbibliography[heading=bibintoc,title={References}]

\newpage
\appendix
\begin{table}[htpb!]
    \centering
    \caption{Welch’s t-test: Average Prices Across Prompt Prefixes}
    \label{tab:welch_test_1}
    \begin{tabular}{lcc}
    \toprule
    & P1 & P2 \\
    \midrule
    Mean Price (last 50 rounds) &  &  \\
    Standard Deviation          &  &  \\
    Observations                &  &  \\
    \midrule
    Difference in Means         & \multicolumn{2}{c}{$^{***}$} \\
    Welch's t-statistic         & \multicolumn{2}{c}{} \\
    Degrees of Freedom          & \multicolumn{2}{c}{} \\
    p-value                     & \multicolumn{2}{c}{<} \\
    \bottomrule
    \multicolumn{3}{p{0.9\linewidth}}{\footnotesize \textit{Notes:} Welch’s t-test compares the mean prices set under each prompt prefix across the 21 experiments, assuming unequal variances. $^{*}$ p$<$0.1, $^{**}$ p$<$0.05, $^{***}$ p$<$0.01. Mean prices are calculated using the final 50 rounds of each experiment.}
    \end{tabular}
\end{table}



\begin{figure}[htpb]
    \centering
    \includesvg[width=1\linewidth]{latex/imgs/res/price_over_time_by_prompt_prefix_combined.svg}
    \caption{Average Price by Prompt and Experiment, shaded areas are 95\% confidence intervals, lines are averaged values out of all 21 runs per prompt (7 runs $\times$ 3 alphas).}
    \label{fig:ts_prices_comb}
\end{figure}

\end{document}   