% Here starts the preamble
\documentclass[12pt]{article}
%%%%%%%%%%%%%%%%%%%%%%%%%%%%%% Doc set %%%%%%%%%%%%%%%%%%%%%%%%%%%%%%%%%%%%
\def\gr{}      % Set the group number cons. across doc
\def\nass{}    % Set the assignment number cons. across doc
\def\cl{Folk Around and Find Out}   % Define the class
%%%%%%%%%%%%%%%%%%%%%%%%%%%%%% Packages %%%%%%%%%%%%%%%%%%%%%%%%%%%%%%%%%%%
\usepackage[utf8]{inputenc} % Unicode characters
% \usepackage[ngerman]{babel} % New German language corrections
\usepackage[T1]{fontenc} % Correct Umlaut appearance
\usepackage{geometry}
 \geometry{
 a4paper,
 total={170mm,257mm},
 left=30mm,
 right=25mm,
 top=30mm,
 bottom=25mm
 }
\usepackage{amsmath}
\usepackage{amssymb}
\usepackage{mathtools}
\usepackage{graphicx}
\graphicspath{latex/imgs/}
\usepackage{pgfplots} % Plots
\pgfplotsset{width=12cm,compat=newest}
\usepackage{xcolor}
\usepackage{multicol}
\usepackage{threeparttable} % For tablenotes environment
\usepackage{transparent}
\usepackage{svg}
\usepackage{url}
\usepackage{booktabs}
\usepackage{tabularx}
\usepackage{subcaption}
\usepackage{float}
%\usepackage{subfig}     % Easy graphics next to each other
\usepackage{enumitem}   % Allows to change enumerate labels
\usepackage{xspace}
\usepackage{fancyhdr}
\pagestyle{fancy}                    % Eigener Seitenstil
\fancyhf{}                           % Alle Kopf- und Fußzeilenfelder bereinigen
\fancyhead[L]{DSDM 2024-2025}            % Kopfzeile links
\fancyhead[C]{\cl}     % Zentrierte Kopfzeile
\fancyhead[R]{BSE}          % Kopfzeile rechts
\renewcommand{\headrulewidth}{0.4pt} % Obere Trennlinie
\fancyfoot[C]{\thepage}              % Seitennummer
\usepackage{pdfpages}
\usepackage{newclude}
\usepackage{hyperref} %Links, z.B. direkt zum gewünschten Kapitel springen
\hypersetup{
    pdftitle    = {\cl - Master thesis \nass},
    pdfsubject  = {This is a submission in the DSDM Masters at BSE.},
    pdfauthor   = {Group\gr},
    % pdfkeywords = {Ha11, Bachelor thesis},
    pdfcreator  = {Overleaf},
    pdfstartview= FitH,        % PDF Fenster ausgefüllt
}
%%%%%%%%%%%%%%%%%%%%%%%%%%%%%%%%% Code highlighting %%%%%%%%%%%%%%%%%%%%%%%%%%%%%%%%%
\usepackage[newfloat]{minted}     % Code highlighting
\newenvironment{code}{\captionsetup{type=listing}}{}
\SetupFloatingEnvironment{listing}{name=Code}

%   Code list in content index
\renewcommand{\listoflistings}{
  \cleardoublepage
  \addcontentsline{toc}{chapter}{List of Code}
  \listof{listing}{List of Code}
}

\setminted{
    %style=monokai,           % Color scheme
    fontsize=\small,         % Text size
    %frame=lines,             % Frame style (none, lines, single)
    %framesep=2mm,           % Frame separation
    %baselinestretch=1.2,    % Line spacing
    breaklines=true,        % Enable line breaking
    linenos=true,           % Show line numbers
    %numbersep=5pt,          % Space between numbers and code
    %tabsize=4,              % Tab size
    autogobble=true,        % Remove common leading whitespace
    %bgcolor=lightgray,      % Background color
    mathescape=true,         % Allow LaTeX math mode in code
    breaklines=true,        % Enable line breaking
    breakanywhere=true,     % Break lines anywhere
    samepage=false          % Explicitly allow page breaks
}
%%%%%%%%%%%%%%%%%%%%%%%%%%%%%%%%%  %%%%%%%%%%%%%%%%%%%%%%%%%%%%%%%%%
\usepackage[font=small,labelfont=bf]{caption}
\usepackage[
    backend=biber, 
    style=apa,
    hyperref=true
    ]{biblatex}
\addbibresource{latex/references.bib}
% Configure cite to show Author (Year)
\DeclareCiteCommand{\cite}
  {\usebibmacro{prenote}}
  {\usebibmacro{citeindex}%
   \printnames{labelname}%
   \space(\printfield{year})}
  {\multicitedelim}
  {\usebibmacro{postnote}}

% Configure parencite to show (Author et al., Year)
%\DeclareNameAlias{labelname}{family-given}
%\renewcommand*{\nameyeardelim}{\addcomma\space}
%\AtEveryCitekey{\ifciteseen{}{\defcounter{maxnames}{1}}}
\usepackage{csquotes}
%\usepackage[all]{hypcap} %Positioning of linked objects after jump
\setlength{\parskip}{0.3\baselineskip} % Adjust spacing between paragraphs
\setlength{\parindent}{0pt} % Remove paragraph indentation if desired
\usepackage{epigraph} % Nicely formatted quotes
\setlength{\epigraphwidth}{0.9\textwidth}
\usepackage[most]{tcolorbox}
% Verfügbar unter: https://github.com/terben/LaTeX_Tutorial_Deutsch/blob/master/Tutorial_18_Arbeiten_mit_grossen_Dokumenten/my_newcommands_german.tex
% Einige nützliche newcommand Definitionen
% für deutsche LaTeX Texte

% Abkürzungen für 'zum Beispiel', 'unter anderem'
% etc. In diesen Abkürzungen ist das Leerzeichen
% zwischen den Buchstaben kleiner als
% normalerweise. Deswegen wird dieses
% mit '\,' anstatt mit einem space gesetzt.
% Abgeschlossen werden die Definitionen durch ein
% \xspace so dass, wenn nötig,
% im Text ein Leerzeichen nach den Makros eingefügt
% wird.
%\newcommand{\zB}{z.\,B.\xspace} % zum Beispiel (z. B.)
%\newcommand{\ua}{u.\,a.\xspace} % unter anderem (u. a.)
%\newcommand{\uU}{u.\,U.\xspace} % unter Umständen (u. U.)

% Kurzformen für existierende, lange LaTeX Befehle:
\newcommand{\tb}{\textbackslash}

% Commands for reference (ref) commands
\newcommand{\chapterref}[1]{Chapter~\ref{#1}}
\newcommand{\sectionref}[1]{Section~\ref{#1}}
\newcommand{\subsectionref}[1]{Subsection~\ref{#1}}
\newcommand{\equationref}[1]{Eq.~\ref{#1}}
\newcommand{\figureref}[1]{Figure~\ref{#1}}
\newcommand{\tableref}[1]{Table~\ref{#1}}
\newcommand{\appref}[1]{Appendix~\ref{#1}}

% Kommandos für Mathemtikkonstrukte:

% Betragsstriche mit korrekter Größe um jedes Objekt:
\newcommand{\abs}[1]{\left|#1\right|}

% Ableitungen mit aufrecht gedruckten Differentialoperator:
\newcommand{\deriv}[2]{\frac{\mathrm{d} #1}{\mathrm{d} #2}}

% Aufrecht gedruckte Eulersche Zahl und imaginäre Einheit:
\newcommand{\euler}{\mathrm{e}}
\newcommand{\imag}{\mathrm{i}}

% Expected value
\DeclareMathOperator{\EX}{\mathbb{E}}% expected value
\newcommand*\mean[1]{\overline{#1}}

% Command for align environment to show transformation steps
\newcommand{\sh}[2]{&& \quad \vert\ \text{#1} #2\\}

% Quotes: enquote and emphasize text for citations
\newcommand{\en}[1]{\enquote{\emph{#1}}}

\begin{document}

% First Cover (for presentation or preliminary cover)
\pdfbookmark[1]{Preliminary Cover}{Preliminary Cover}
\begin{titlepage}
    \centering
    \includesvg[width=0.75\textwidth]{latex/imgs/BSE Barcelona Graduate School of Economics.svg}\par
    \vspace{1.5cm}

    {\LARGE\bfseries Master's Degree in Data Science\par}
    \vspace{0.3cm}
    {\LARGE Data Science for Decision Making Programme\par}

    \vfill

    {\huge\bfseries \en{Folk Around and Find Out: Algorithmic Collusion and the Limits of Coordination}\par}

    \vfill

    {\Large Authors:\par}
    {\Large Moritz Peist, Julián Romero, Lucia Sauer\par}
    \vspace{0.5cm}
    {\Large Supervisors:\par}
    {\Large Christopher Rauh, Hannes Mueller\par}

    \vfill

    {\Large July 2025\par}
\end{titlepage}

% Mandatory for abstract
\pdfbookmark[1]{Abstract}{abstract}
\begin{titlepage}
    % Abstract title styling
    {\LARGE\bfseries Abstract in English\par}
    \vspace{0.5cm}
    The Folk Theorem establishes that collusion can be sustained in repeated interactions, yet empirical evidence suggests coordination becomes more difficult as market participants increase. This thesis presents the first test of whether Large Language Model (LLM) agents exhibit this pattern. In controlled experiments with 2--5 competing agents, we find LLM coordination erodes predictably with competition. Our results show a 3.7\% reduction in equilibrium price for each additional firm (p < 0.001), with prices declining smoothly. This culminates in a 10.6\% total price reduction from duopoly to five-agent markets, providing quantitative evidence on algorithmic collusion boundaries in the AI era.

    \vspace{1.5cm}

    {\LARGE\bfseries Abstract in Spanish\par}
    \vspace{0.5cm}
    El Teorema de Folk establece que la colusión puede mantenerse en interacciones repetidas, pero la evidencia empírica sugiere que la coordinación se vuelve más difícil a más participantes en el mercado. Esta tesis presenta una primera prueba  de si agentes de modelos de lenguaje grandes (LLM) muestran este patrón. En experimentos controlados con 2-5 agentes competidores, encontramos que la coordinación entre LLM se erosiona predeciblemente con la competencia. Resultados muestran una reducción del 3,7\% en el precio de equilibrio por empresa adicional (p < 0,001), con disminución precios. Esto culmina en reducción total del 10,6\% desde duopolio hasta mercados con cinco agentes, proporcionando evidencia cuantitativa sobre límites colusión algorítmica en era de la IA. 

    \vfill

    {\large \textbf{Keywords in English}: algorithmic collusion, Folk Theorem, LLM agents.\par}
    
    \vspace{0.5cm}

    {\large \textbf{Keywords in Spanish}: colusión algorítmica, Teorema de Folk, agentes LLM.\par}
    \vspace{2cm}

\end{titlepage}

\pdfbookmark[1]{Title page}{Title page}
\begin{titlepage}
	\centering
	\includesvg[width=0.75\textwidth]{latex/imgs/BSE Barcelona Graduate School of Economics.svg}\par\vspace{1cm}
	{\huge\bfseries Folk Around and Find Out:\par}
    {\large\bfseries Algorithmic Collusion and the Limits of Coordination\par}
	\vspace{1cm}
    \noindent\rule{\textwidth}{1pt}
    {\Large Moritz Peist (254017)\par}
    {\Large Julián Romero (253764)\par}
    {\Large Lucia Sauer (254053)\par}
    \noindent\rule{\textwidth}{1pt}
    \vfill
    \begin{abstract}
        \noindent
        The \emph{Folk Theorem} establishes that collusion can be sustained in repeated interactions, yet empirical evidence suggests coordination becomes more difficult as market participants increase. This thesis presents the first test of whether Large Language Model (LLM) agents exhibit this pattern. In controlled experiments with 2-5 competing agents, we find LLM coordination erodes predictably with competition. Our results show a 3.7\% reduction in equilibrium price for each additional firm (p < 0.001), with prices declining smoothly. This culminates in a 10.6\% total price reduction from duopoly to five-agent markets, providing quantitative evidence on algorithmic collusion boundaries in the AI era.
    \end{abstract}
	\vfill
    % Bottom of the page
	{\large \today\par}
\end{titlepage}

\tableofcontents
\thispagestyle{empty}

\newpage
\addtocounter{page}{-1}
\include*{latex/chapters/part1}
\include*{latex/chapters/part2}
\include*{latex/chapters/part3}
\include*{latex/chapters/part4}
\include*{latex/chapters/part5}
\include*{latex/chapters/part6}

\newpage
\printbibliography[heading=bibintoc,title={References}]

\newpage
\appendix
\begin{table}[htpb!]
    \centering
    \caption{Welch’s t-test: Average Prices Across Prompt Prefixes}
    \label{tab:welch_test_1}
    \begin{tabular}{lcc}
    \toprule
    & P1 & P2 \\
    \midrule
    Mean Price (last 50 rounds) &  &  \\
    Standard Deviation          &  &  \\
    Observations                &  &  \\
    \midrule
    Difference in Means         & \multicolumn{2}{c}{$^{***}$} \\
    Welch's t-statistic         & \multicolumn{2}{c}{} \\
    Degrees of Freedom          & \multicolumn{2}{c}{} \\
    p-value                     & \multicolumn{2}{c}{<} \\
    \bottomrule
    \multicolumn{3}{p{0.9\linewidth}}{\footnotesize \textit{Notes:} Welch’s t-test compares the mean prices set under each prompt prefix across the 21 experiments, assuming unequal variances. $^{*}$ p$<$0.1, $^{**}$ p$<$0.05, $^{***}$ p$<$0.01. Mean prices are calculated using the final 50 rounds of each experiment.}
    \end{tabular}
\end{table}


\end{document}   